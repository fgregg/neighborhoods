\documentclass[12pt,draft,letter]{article}
\begin{document}
\section{Introduction}
Social scientists have collected a wide and sophisticated arsenal of
ways to group similar elements of network. However, except in simplest
cases, what determines the output is the simple, iterative algorithm
of the researcher fiddling with weights until the groupings look how
she wants them to. Our methods to group similar elements depends on us
knowing how to measure similarity, and we usually don't.

As long as a measure of similarity is binary, i.e. did ego nominate
alter as friend, life is easy. But, as soon as a variable has more
than two possible values then the researcher must decide how to turn
those values into similarities. Different, reasonable choices will
lead to different groupings. A careful researcher may check
her choices for robustness, but may only learn that her errors are
robust. The difficulties grow combinatorially if we decide that
a second, third, or more variable might be of some interest.

In the end, we adjust the various weights until the groupings come out
looking about right. We would have been better off just drawing what
we wanted in the first place.

We would be better off because we humans are better apprehending an
assortment of things as a group than we are at assigning importance to
variables by manually assigning weights, which we are very, very bad
at.

While the interestingness of apprehended unities may vary 
unfairly between individual sociologists (not every one has Weberian gift),
we are professionally required to be interested in the way that a
collection of people groups themselves.

If we can collect information on how people group themselves, then we
can learn what counts for similarity for these people. Understanding
what similarity means for is not only scientifically interesting in
itself, it will also allow for principles prediction of how similar
collections of people will group themselves.

In this paper I discuss how to do learn about similarity using the
case of city neighborhoods. 
\section{Neighborhood Model Representation}

\section{Mathematics}

\section{Results}

\section{Conclusion}

\end{document}
