\documentclass[12pt,draft,letter]{article}
\usepackage{tikz}
\usepackage{amsmath}
\usepackage{adjustbox}

\usetikzlibrary{arrows}
\usetikzlibrary{bayesnet}

\DeclareMathOperator*{\argmin}{\arg\!\min}
\DeclareMathOperator*{\argmax}{\arg\!\max}


\begin{document}

\section*{Introduction}
Social scientists have collected a wide and sophisticated arsenal of
ways to group similar elements of network. However, except in simplest
cases, what determines the output is the simple, iterative algorithm
of the researcher fiddling with weights until the groupings look how
she wants them to. Our methods to group similar elements depends on us
knowing what similarity means in our network. We usually don't.

When our similarity is binary, i.e. did ego nominate
alter as friend, life is easy. But, as soon as a variable has more
than two possible values then the researcher must decide how to turn
those values into similarities. Different, reasonable choices will
lead to different groupings. A careful researcher may check
her choices for robustness, but may only learn that her errors are
robust. The difficulties grow combinatorially if we decide that
a second, third, or more variable might be of some interest.

In the end, we adjust the various weights until the groupings come out
looking about right. We would have been better off just drawing what
we wanted in the first place.

We would be better off because we humans are all too good at
apprehending an assortment of things as a group than we are at
assigning importance to variables by manually assigning weights, which
we are very, very bad at.

While the interestingness of apprehended unities may vary 
unfairly between individual sociologists (not every one has Weberian gift),
we are professionally required to be interested in the way that a
collection of people groups themselves.

If we can collect information on how people group themselves, then we
can learn what counts for similarity for these people. Understanding
what similarity means for is not only scientifically interesting in
itself, it will also allow for principles prediction of how similar
collections of people will group themselves.

In this paper I discuss how to do learn about similarity using the
case of city neighborhoods. 

\section*{Neighborhood Model Representation}
Let's image a very small city consisting of four blocks. We can
represent this tiny town as a network where each block is connected
to the blocks that share a street segment as a block face. In this
representation, blocks that are kitty corner are not directly
connected. We'll index the blocks as $1$, $2$, $3$, and $4$.

\begin{figure}
\centering
\tikz{
\draw[help lines] (0,0) grid (2,2);
\node at (0.5, 0.5) {3} ;
\node at (1.5, 1.5) {2} ;
\node at (0.5, 1.5) {1} ;
\node at (1.5, 0.5) {4} ;
}
\end{figure}

\begin{figure}
\centering

\tikz{ %
  \node[latent] (1) {$1$} ; %
  \node[latent, below left=of 1] (2) {$2$} ; %
  \node[latent, below right=of 1] (3) {$3$} ; %
  \node[latent, below left=of 3] (4) {$4$} ; %
  \edge[-] {2,3} {1} ; %
  \edge[-] {2,3} {4} ; %
}

\end{figure}

In our city, there are two neighborhoods. Each block belongs to either
one or the other of these neighborhoods. Neighboring blocks that are
similar are more apt to belong to the same neighborhood and
neighboring blocks that are different are more apt to belong to
different neighborhoods.

We'll denote the neighborhood that the $i$th block belongs to as
$y_i$, and the similarity between blocks $i$ and $j$ as $\phi_{i,j}$.

\begin{figure}[!h]
\centering

\tikz{ %
  \node[latent] (1) {$y_1$} ; %
  \node[latent, below left=of 1] (2) {$y_2$} ; %
  \node[latent, below right=of 1] (3) {$y_3$} ; %
  \node[latent, below left=of 3] (4) {$y_4$} ; %
  \factor[below left=of 1] {1-2} {$\phi_{1,2}$} {} {} ;
  \factor[below right=of 1] {1-3} {$\phi_{1,3}$} {} {} ;
  \factor[below right=of 2] {2-4} {$\phi_{2,4}$} {} {} ;
  \factor[below left=of 3] {3-4} {$\phi_{3,4}$} {} {} ;
  \factoredge[-] {1} {1-2} {2} ; %
  \factoredge[-] {1} {1-3} {3} ; %
  \factoredge[-] {2} {2-4} {4} ; %
  \factoredge[-] {3} {3-4} {4} ; %
  %\edge[-] {2,3} {4} ; %
}

\end{figure}

We want similar, neighboring blocks to belong to the same
neighborhood. One way to formalize this is is to score every possible
assignment of neighborhoods in such a way that our preferred patterns
have the best score.

Let's call a pattern of assignment of blocks to neighborhoods as
$\mathbf{y}$.  The score of $\mathbf{y}$ will be
$\operatorname{E}(\mathbf{y})$ which will take the following form:

\begin{align}
\operatorname{E}(\mathbf{y}) = \sum_{<i j>}^{\mathcal{N}}\epsilon_{i,j}(y_i,y_j,\phi_{i,j})
\end{align}

Where $\mathcal{N}$ is the set of 2-tuples of indices of neighboring
blocks and the index of the first block is smaller than the index of
the second block. Also, where

\begin{equation}
\epsilon_{i,j}(y_i,y_j\phi_{i,j}) = \begin{cases}
  0 \quad\quad y_i = y_j \\
  \phi_{i,j} \quad y_i \neq y_j
\end{cases}
\end{equation}

Let's say that we want the lowest score. If we have our similarity
measures $\phi_{i,j}$ be positive when blocks are similar and negative
when blocks are different we will encourage neighboring blocks to
belong to the same neighborhood. Let's say that our city has the
following $\phi$'s.

\begin{align*}
&\phi_{1,2} = 1 \\
&\phi_{1,3} = -1 \\
&\phi_{2,4} = -1 \\
&\phi_{3,4} = 1
\end{align} 

Then, the lowest energy assignments are the assignments that put
blocks $1$ and $2$ in one neighborhood and $3$ and $4$ in the other
neighborhood.

\begin{figure}[!h]
\centering

\tikz{ %
  \node[latent] (1) {$y_1$} ; %
  \node[latent, below left=of 1] (2) {$y_2$} ; %
  \node[latent, below right=of 1] (3) {$y_3$} ; %
  \node[latent, below left=of 3] (4) {$y_4$} ; %
  \factor[below left=of 1] {1-2} {$1$} {} {} ;
  \factor[below right=of 1] {1-3} {$-1$} {} {} ;
  \factor[below right=of 2] {2-4} {$-1$} {} {} ;
  \factor[below left=of 3] {3-4} {$1$} {} {} ;
  \factoredge[-] {1} {1-2} {2} ; %
  \factoredge[-] {1} {1-3} {3} ; %
  \factoredge[-] {2} {2-4} {4} ; %
  \factoredge[-] {3} {3-4} {4} ; %
  %\edge[-] {2,3} {4} ; %
}

\end{figure}

\input{energy_table.tex}

In general, for a given set of similarity measures between blocks
there are one or more assignment that have a lowest score. For small
networks, we can find these best assignments by checking all possible
patterns of assignments. Unfortunately, this becomes quickly
impractical for larger cities. With two possible neighborhoods, the
number of possible assignments is $2^N$ where N is the number of
blocks.

However, even if we have many blocks, if we have similarities
between the blocks we can find a lowest scoring assignment
approximately using the QPBO algorithm.\footnote{Minimizing
  non-submodular functions with graph cuts – a review (are there any
  bounds worth mentioning}. This will be true for every scoring
function that decomposes over blocks and edges, i.e. has the following
form:

\begin{align}
\operatorname{E}(\mathbf{y}) = \sum_i^N\epsilon_i(y_i) + \sum_{<i j>}^{\mathcal{N}}\epsilon_{i,j}(y_i,y_j)
\end{align}



As we'll see, this will allow us to go in
the opposite direction. If we know the correct assignment, we can find
similarities.

\section*{Learning similarities}
Let's consider similarity measures that take the following linear form

\begin{align}
\phi_{i,j} = w_0 + w_1\operatorname{S_1}(x_{i,1}, x_{j,1}) +
w_2\operatorname{S_2}(x_{i,2}, x_{j,2}) + ... +
w_n\operatorname{S_z}(x_{i,m},x_{j,m})
\end{align} 

Where $x_{i,k}$ is the $k$th feature of block $i$. These
features can be scalars or vectors of numbers and could correspond to
attributes like population or distribution of race and ethnicity. A
function $\operatorname{S}$ maps two features to some scalar. These
will typically be similarity measures like the Jensen Shannon
divergence or the absolute value. For each $S$ and features $a$ and
$b$, $\operatorname{S}(a,b) = \operatorname{S}(b,a)$.

With this choice of form, learning the similarities between blocks
means finding the set of weights $w_0, w_1, ..., w_n$ that produce
similarities $\phi$'s that give a particular, target neighborhood
assignment the lowest score of all possible ways of assigning blocks
to neighborhoods. 

However, this problem does not have a unique solution. In the above
four block assignment, the lowest score assignments would have the
lowest scores if the similarities were 2, -2, -2, 2 or 1000, -1000,
-1000, 1000. So, we will say that we want the lowest scoring
assignment to not just have the lowest score, but we want to find the
weights that give the target assignment a lower score to the next
lowest scoring assignment by the widest possible margin. We will also
have to add some additional constraint on $w$. We can do that by
requiring that $||\mathbf{w}||=1$.

So, now we can state the problem as 
%
\begin{align*}
&\argmax_{\mathbf{w}:||\mathbf{w}||=1} \mathbf{\gamma} \\
&\text{such that} \\
&\operatorname{E}(\mathbf{y}, \mathbf{s}, \mathbf{w})
- \operatorname{E}(\mathbf{y}^*, \mathbf{s}, \mathbf{w}) \geq \gamma\\ 
&\text{for all } \mathbf{y} \text{ where } \mathbf{y} \text{ is in the set of
  possible neighborhood assignments}\\
&\text{and } \mathbf{y} \neq \mathbf{y}^*
\end{align*}
%

Here $\mathbf{y}*$ is the target neighborhood assignment and \mathbf{s}
are similarity measures between blocks. 


In turns out, in a way that I need to better understand that this is
equivalent to the following quadratic program.
%
\begin{align*}
&\argmin_{\mathbf{w}} \frac{1}{2}||\mathbf{w}||^2 \\
&\text{such that} \\
&\operatorname{E}(\mathbf{y}, \mathbf{s}, \mathbf{w})
- \operatorname{E}(\mathbf{y}^*, \mathbf{s}, \mathbf{w}) \geq 1 \\ 
&\text{for all } \mathbf{y} \text{ where } \mathbf{y} \text{ is in the set of
  possible neighborhood assignments}\\
&\text{and } \mathbf{y} \neq \mathbf{y}^*
\end{align*}


\subsection{Learning Sketch}
We usually can't directly solve that quadratic program because there
are so many possible assignments , however we can still find the
optimal weights through the following procedure.

Initialize the weights to some starting value. Create an empty set of
constraints $\mathcal{S}$. Then, find the neighborhood assignment that
has the lowest score given those weights. Call this assignment
$\mathbf{y}_1$ and add it to the constraint set $\mathcal{S}$.

Update the weights by solving the quadratic program: 
%
\begin{align*}
&\argmin_{\mathbf{w}} \frac{1}{2}||\mathbf{w}||^2 \\
&\text{such that} \\
&\operatorname{E}(\mathbf{y}, \mathbf{s}, \mathbf{w})
- \operatorname{E}(\mathbf{y}^*, \mathbf{s}, \mathbf{w}) \geq 1 \\ 
&\text{for all } \mathbf{y} \text{ where } \mathbf{y} \text{ is in } \mathcal{S}\\
&\text{and } \mathbf{y} \neq \mathbf{y}^*
\end{align*}
%

Now find the lowest energy neighborhood assignment given the updated
weights. Call this assignment $\mathbf{y}_2$ and add it to the
constraint set $\mathcal{S}$. Continue until the either the lowest
energy assignment given the weights is either the target assignment.

See Szummer, Learning CRFs using Graph Cuts for a more thorough
explanation.

\subsection*{Relaxations}
Some readers may have noticed that the our four-city block example
doesn't have a solution for the learning problem we have set up. There
are two different neighborhood assignments that have the same lowest
score, and they both achieve our goal of putting similar blocks in the 
same neighborhood and dissimilar blocks in different neighborhoods.

One way to see why this happens is to condense the four block network
into a two neighborhood graph. 

\begin{tabular}{cc}
  \adjustbox{valign=c}{
  \tikz{ %
    \node[latent] (1) {$1$} ; %
    \node[latent, below left=of 1] (2) {$2$} ; %
    \node[latent, below right=of 1] (3) {$3$} ; %
    \node[latent, below left=of 3] (4) {$4$} ; %
    \node[rectangle, rotate fit=45, draw=red, fit= (1) (2), text=red] {A};
    \node[rectangle, draw=red, rotate fit=45, fit= (3) (4), text=red] {B};
    \edge[-] {2,3} {1} ; %
    \edge[-] {2,3} {4} ; %
  }}
  &
  \begin{adjustbox}{valign=c}
  \tikz{ %
    \node[latent, draw=red, text=red] (A) {A} ; %
    \node[latent, draw=red, text=red, below right=of A] (B) {B} ; %
    \edge[-] {A} {B} ; %
  }
  \end{adjustbox}
\end{tabular}

To distinguish the two neighborhoods, we want to give each
neighborhood a different color. In particular, we want to color the
neighborhoods so that no two adjacent neighborhoods have the same
color. We'll call such a color assignment a coloring. 

With two colors there are two ways to color our neighborhood.

\begin{tabular}{cc}
  \tikz{ %
    \node[latent, draw=red, text=red, fill=black] (A) {A} ; %
    \node[latent, draw=red, text=red, below right=of A] (B) {B} ; %
    \edge[-] {A} {B} ; %
  }
  &
  \tikz{ %
    \node[latent, draw=red, text=red] (A) {A} ; %
    \node[latent, draw=red, text=red, below right=of A, fill=black] (B) {B} ; %
    \edge[-] {A} {B} ; %
  }
\end{tabular}

If we had more than two neighborhoods, we might need more than two
colors. Whether you can color the neighborhoods with a certain number
of colors depends upon the number of neighborhood and how they are
connected. The same is true for the number of different ways you can
color neighborhoods with a given colors.\footnote{Every graph has a 
chromatic polynomial that determines the number of possible
n-colorings of vertices in the graph}

For twenty or so city neighborhood, we will usually need four colors,
and the number of different ways to four-color those neighborhoods
will be well over a thousand. That means in a larger city, there are
thousands of patterns of assigning neighborhood labels to blocks that 
will put the same blocks together, and therefore have the same score.

In theory, we could overcome this multiplicity of equivalent
assignments by relaxing our learning objective. Instead of looking for
the weights that makes our target assignment the lowest scoring
assignment, we can look for the weights that makes our target
assignment lower scoring than as many assignments as we'd like. 

\begin{align*}
&\argmin_{\mathbf{w}} \frac{1}{2}||\mathbf{w}||^2 + C\cdot\xi\\
&\text{such that} \\
&\operatorname{E}(\mathbf{y}, \mathbf{s}, \mathbf{w})
- \operatorname{E}(\mathbf{y}^*, \mathbf{s}, \mathbf{w}) \geq 1 - \xi\\ 
&\text{for all } \mathbf{y} \text{ where } \mathbf{y} \text{ is in the set of
  possible neighborhood assignments}\\
&\text{and } \mathbf{y} \neq \mathbf{y}^*
\end{align*}

This `slack' formulation validly specifies that we seek the weights
that make target assignment and it's equivalent cousins the best
solution. Unfortunately, in practice, the good performance in out of
set prediction seems to depends upon incorporating an empirical loss
function into the objective function, like below
%
\begin{align*}
&\argmin_{\mathbf{w}} \frac{1}{2}||\mathbf{w}||^2 +
  C\cdot\xi\\
&\text{such that} \\
&\operatorname{E}(\mathbf{y}, \mathbf{s}, \mathbf{w})
- \operatorname{E}(\mathbf{y}^*, \mathbf{s}, \mathbf{w}) \geq \Delta(\mathbf{y}^*, \mathbf{y}) - \xi\\ 
&\text{for all } \mathbf{y} \text{ where } \mathbf{y} \text{ is in the set of
  possible neighborhood assignments}\\
&\text{and where } \Delta \text{ is an empirical loss.}\\
\end{align*}
%

The key to solving this problem is finding assignments that have the
lowest score, where the score incorporates the empirical loss: i.e.
%
\begin{align}
\operatorname{E}(\mathbf{y}) = \sum_i^N\epsilon_i(y_i) + \sum_{<i
  j>}^{\mathcal{N}}\epsilon_{i,j}(y_i,y_j) + C\cdot\Delta(y*, y)
\end{align}
%
If the loss can be decomposed over the blocks like
%
\begin{align}
\operatorname{E}(\mathbf{y}) = \sum_i^N\epsilon_i(y_i) + \sum_{<i
  j>}^{\mathcal{N}}\epsilon_{i,j}(y_i,y_j) + C\cdot\sum_i^N\Delta(y_i*, y_i)
\end{align}
%
then the objective has the form that allows us to use an algorithm
like QPBO to find a quick minimum assignment. If the loss function does not
decompose we have to check an exponentially many possible assignment,
typically an intractable problem.

Among the loss functions that do decompose over blocks, a common
choice is the Hamming loss, $\sum_i^N\begin{cases}
  0 \quad y_i^* = y_i \\
  1 \quad y_i^* \neq y_i
\end{cases}$. This loss, and its cousins, are completely inappropriate
for our problem, as we have currently posed it.

We want a loss function that increases the further we are from our
target assignment. If two assignments group the same sets of blocks
together we would like the hamming loss to be 0. However, with the
Hamming loss, the assignments with the highest possible
losses are assignments that assign the same blocks to the same
neighborhoods as our target assignment, but assigns different labels
to those neighborhoods. 

There are loss functions that behave as we would like, like the Rand
Index. However, I know of no such loss function that decomposes over
blocks or edges. Without that property, the learning problem is
computationally intractable. Under some restricted circumstances, the
higher order loss functions can be used by introducing auxillary
variables into the problem, but these approaches are computationally
infeasible for cities with realistic number of
neighborhoods. {citations here}

The ultimate impediment is that for every way of labeling blocks into
neighborhoods, there are large number of equivalent groupings that
differ only in the value of the labels, and they each have the same
score. So, we might proceed, by arbitrarily giving one of those
equivalent labelings priority.

One way to do that is to change our energy function to include seeds. 
%
\begin{align}
\operatorname{E}(\mathbf{y}) = \sum_k^{\mathcal{S}}\epsilon_k(y_k) + \sum_{<i
  j>}^{\mathcal{N}}\epsilon_{i,j}(y_i,y_j,w)
\end{align}
%
Where $\mathcal{S}$ is a subset of all the blocks, and 
%
\begin{align}
\epsilon_k(y_k) = \begin{cases}
  0 \quad y_k^* = y_k \\
  P \quad y_i^* \neq y_j
\end{cases}
\end{align}

But how do we determine the subset of blocks and P. If the we assigned
this cost to every block, then this would be equivalent to 

\begin{align*}
&\argmin_{\mathbf{w}} \frac{1}{2}||\mathbf{w}||^2 +
  C\cdot\xi\\
&\text{such that} \\
&\sum_{<i,j>}^{\mathcal{N}}\epsilon_{i,j}(y_i,y_j)
   - \sum_{<i,j>}^{\mathcal{N}}\epsilon_{i,j}(y_i^*,y_j^*)  \geq (1-P)\cdot\sum_i^N\Delta(y_i*, y_i) - \xi\\ 
&\text{for all } \mathbf{y} \text{ where } \mathbf{y} \text{ is in the set of
  possible neighborhood assignments}\\
&\text{and where } \Delta \text{ is an empirical loss.}\\
\end{align*}


Notice that the contributions of the block scores does not depend upon
w. 


Ideally, we would 




neighborhoods there are large nu
















admits a quick, approximi

Which is of the form that can be quickly approxmately solved by QPBO. 





Frequently, the hamming loss is the number of
differences in block assignments between an assignment and a target
assignment.  

A key property of the hamming loss is that it decomposes over
blocks. That is we can 




Unfortunately

However, even here 



(idea of seeding constraints with equivalent graphs)

In practice, however it does not work. 


However, in practice the learning will be much too slow. The learning
algorithm may spend a great deal of time just finding equivalent 

make family of equivalent  


A multiplicity of solutions is not really such a problem,
(hmm... could we 






. These graphs are 



\section*{Mathematics}

\section*{Results}

\section*{Conclusion}

\end{document}
