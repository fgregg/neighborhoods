\documentclass[12pt,draft,letter]{article}
\usepackage{tikz}
\usepackage{amsmath}

\usetikzlibrary{arrows}
\usetikzlibrary{bayesnet}

\begin{document}

\section*{Introduction}
Social scientists have collected a wide and sophisticated arsenal of
ways to group similar elements of network. However, except in simplest
cases, what determines the output is the simple, iterative algorithm
of the researcher fiddling with weights until the groupings look how
she wants them to. Our methods to group similar elements depends on us
knowing what similarity means in our network. We usually don't.

When our similarity is binary, i.e. did ego nominate
alter as friend, life is easy. But, as soon as a variable has more
than two possible values then the researcher must decide how to turn
those values into similarities. Different, reasonable choices will
lead to different groupings. A careful researcher may check
her choices for robustness, but may only learn that her errors are
robust. The difficulties grow combinatorially if we decide that
a second, third, or more variable might be of some interest.

In the end, we adjust the various weights until the groupings come out
looking about right. We would have been better off just drawing what
we wanted in the first place.

We would be better off because we humans are all too good at
apprehending an assortment of things as a group than we are at
assigning importance to variables by manually assigning weights, which
we are very, very bad at.

While the interestingness of apprehended unities may vary 
unfairly between individual sociologists (not every one has Weberian gift),
we are professionally required to be interested in the way that a
collection of people groups themselves.

If we can collect information on how people group themselves, then we
can learn what counts for similarity for these people. Understanding
what similarity means for is not only scientifically interesting in
itself, it will also allow for principles prediction of how similar
collections of people will group themselves.

In this paper I discuss how to do learn about similarity using the
case of city neighborhoods. 

\section*{Neighborhood Model Representation}
Let's image a very small city consisting of four blocks. We can
represent this tiny town as a network where each block is connected
to the blocks that share a street segment as a block face. In this
representation, blocks that are kitty corner are not directly
connected. We'll index the blocks as $1$, $2$, $3$, and $4$.

\begin{figure}
\centering
\tikz{
\draw[help lines] (0,0) grid (2,2);
\node at (0.5, 0.5) {3} ;
\node at (1.5, 1.5) {2} ;
\node at (0.5, 1.5) {1} ;
\node at (1.5, 0.5) {4} ;
}
\end{figure}

\begin{figure}
\centering

\tikz{ %
  \node[latent] (1) {$1$} ; %
  \node[latent, below left=of 1] (2) {$2$} ; %
  \node[latent, below right=of 1] (3) {$3$} ; %
  \node[latent, below left=of 3] (4) {$4$} ; %
  \edge[-] {2,3} {1} ; %
  \edge[-] {2,3} {4} ; %
}

\end{figure}

In our city, there are two neighborhoods. Each block belongs to either
one or the other of these neighborhoods. Neighboring blocks that are
similar are more apt to belong to the same neighborhood and
neighboring blocks that are different are more apt to belong to
different neighborhoods.

We'll denote the neighborhood that the $i$th block belongs to as
$y_i$, and the similarity between blocks $i$ and $j$ as $\phi_{i,j}$.

\begin{figure}[!h]
\centering

\tikz{ %
  \node[latent] (1) {$y_1$} ; %
  \node[latent, below left=of 1] (2) {$y_2$} ; %
  \node[latent, below right=of 1] (3) {$y_3$} ; %
  \node[latent, below left=of 3] (4) {$y_4$} ; %
  \factor[below left=of 1] {1-2} {$\phi_{1,2}$} {} {} ;
  \factor[below right=of 1] {1-3} {$\phi_{1,3}$} {} {} ;
  \factor[below right=of 2] {2-4} {$\phi_{2,4}$} {} {} ;
  \factor[below left=of 3] {3-4} {$\phi_{3,4}$} {} {} ;
  \factoredge[-] {1} {1-2} {2} ; %
  \factoredge[-] {1} {1-3} {3} ; %
  \factoredge[-] {2} {2-4} {4} ; %
  \factoredge[-] {3} {3-4} {4} ; %
  %\edge[-] {2,3} {4} ; %
}

\end{figure}

We want similar, neighboring blocks to belong to the same
neighborhood. One way to formalize this is is to score every possible
assignment of neighborhoods in such a way that our preferred patterns
have the best score.

Let's call a pattern of assignment of blocks to neighborhoods as $Y$.
The score of $Y$ will be $\operatorname{E}(Y)$ which will take the
following form:

\begin{align}
\operatorname{E}(Y) = \sum_{<i j>}^{\mathcal{N}}\epsilon_{i,j}(y_i,y_j,\phi_{i,j})
\end{align}

Where $\mathcal{N}$ is the set of 2-tuples of indices of neighboring
blocks and the index of the first block is smaller than the index of
the second block. Also, where

\begin{equation}
\epsilon_{i,j}(y_i,y_j\phi_{i,j}) = \begin{cases}
  0 \quad\quad y_i = y_j \\
  \phi_{i,j} \quad y_i \neq y_j
\end{cases}
\end{equation}

Let's say that we want the lowest score. If we have $\phi_{i,j}$ be
positive when blocks are similar and negative when blocks are
different we will encourage neighboring blocks to belong to the same
neighborhood. Let's say that our city has the following $\phi$'s.

\begin{align*}
&\phi_{1,2} = 1 \\
&\phi_{1,3} = -1 \\
&\phi_{2,4} = -1 \\
&\phi_{3,4} = 1
\end{align} 

Then, the lowest energy assignments are the assignments that put
blocks $1$ and $2$ in one neighborhood and $3$ and $4$ in the other
neighborhood.

\begin{figure}[!h]
\centering

\tikz{ %
  \node[latent] (1) {$y_1$} ; %
  \node[latent, below left=of 1] (2) {$y_2$} ; %
  \node[latent, below right=of 1] (3) {$y_3$} ; %
  \node[latent, below left=of 3] (4) {$y_4$} ; %
  \factor[below left=of 1] {1-2} {$1$} {} {} ;
  \factor[below right=of 1] {1-3} {$-1$} {} {} ;
  \factor[below right=of 2] {2-4} {$-1$} {} {} ;
  \factor[below left=of 3] {3-4} {$1$} {} {} ;
  \factoredge[-] {1} {1-2} {2} ; %
  \factoredge[-] {1} {1-3} {3} ; %
  \factoredge[-] {2} {2-4} {4} ; %
  \factoredge[-] {3} {3-4} {4} ; %
  %\edge[-] {2,3} {4} ; %
}

\end{figure}

\begin{tabular}{cl}
  \\
  $\operatorname{E}(\mathbf{y}) = -2$
  & 
  \begin{tabular}{cccc}
    \scalebox{0.5}{
      \tikz{ %
        \node[latent] (1) {$0$} ; %
        \node[latent, below left=of 1] (2) {$0$} ; %
        \node[latent, fill=black, below right=of 1] (3) {\textcolor{white}{$1$}} ; %
        \node[latent, fill=black, below left=of 3] (4) {\textcolor{white}{$1$}} ; %
        \factor[below left=of 1] {1-2} {} {} {} ;
        \factor[below right=of 1] {1-3} {} {} {} ;
        \factor[below right=of 2] {2-4} {} {} {} ;
        \factor[below left=of 3] {3-4} {} {} {} ;
        \factoredge[-] {1} {1-2} {2} ; %
        \factoredge[-] {1} {1-3} {3} ; %
        \factoredge[-] {2} {2-4} {4} ; %
        \factoredge[-] {3} {3-4} {4} ; %
        %\edge[-] {2,3} {4} ; %
      } 
    }
    &
    \scalebox{0.5}{
      \tikz{ %
        \node[latent, fill=black] (1) {\textcolor{white}{$8$}} ; %
        \node[latent, fill=black, below left=of 1] (2) {\textcolor{white}{$1$}} ; %
        \node[latent, below right=of 1] (3) {$0$} ; %
        \node[latent, below left=of 3] (4) {$0$} ; %
        \factor[below left=of 1] {1-2} {} {} {} ;
        \factor[below right=of 1] {1-3} {} {} {} ;
        \factor[below right=of 2] {2-4} {} {} {} ;
        \factor[below left=of 3] {3-4} {} {} {} ;
        \factoredge[-] {1} {1-2} {2} ; %
        \factoredge[-] {1} {1-3} {3} ; %
        \factoredge[-] {2} {2-4} {4} ; %
        \factoredge[-] {3} {3-4} {4} ; %
        %\edge[-] {2,3} {4} ; %
      } 
    }
    \\
  \end{tabular}
  \\ \hline

  $\operatorname{E}(\mathbf{y}) = 0$
  &
  \begin{tabular}{cccc}
    \scalebox{0.5}{
      \tikz{ %
        \node[latent, fill=black] (1) {\textcolor{white}{$1$}} ; %
        \node[latent, fill=black, below left=of 1] (2) {\textcolor{white}{$1$}} ; %
        \node[latent, fill=black, below right=of 1] (3) {\textcolor{white}{$1$}} ; %
        \node[latent, fill=black, below left=of 3] (4) {\textcolor{white}{$1$}} ; %
        \factor[below left=of 1] {1-2} {} {} {} ;
        \factor[below right=of 1] {1-3} {} {} {} ;
        \factor[below right=of 2] {2-4} {} {} {} ;
        \factor[below left=of 3] {3-4} {} {} {} ;
        \factoredge[-] {1} {1-2} {2} ; %
        \factoredge[-] {1} {1-3} {3} ; %
        \factoredge[-] {2} {2-4} {4} ; %
        \factoredge[-] {3} {3-4} {4} ; %
        %\edge[-] {2,3} {4} ; %
      } 
    }
    \\
    \scalebox{0.5}{
      \tikz{ %
        \node[latent] (1) {$6$} ; %
        \node[latent, fill=black, below left=of 1] (2) {\textcolor{white}{$1$}} ; %
        \node[latent, fill=black, below right=of 1] (3) {\textcolor{white}{$1$}} ; %
        \node[latent, fill=black, below left=of 3] (4) {\textcolor{white}{$1$}} ; %
        \factor[below left=of 1] {1-2} {} {} {} ;
        \factor[below right=of 1] {1-3} {} {} {} ;
        \factor[below right=of 2] {2-4} {} {} {} ;
        \factor[below left=of 3] {3-4} {} {} {} ;
        \factoredge[-] {1} {1-2} {2} ; %
        \factoredge[-] {1} {1-3} {3} ; %
        \factoredge[-] {2} {2-4} {4} ; %
        \factoredge[-] {3} {3-4} {4} ; %
        %\edge[-] {2,3} {4} ; %
      } 
    }
    &
    \scalebox{0.5}{
      \tikz{ %
        \node[latent, fill=black] (1) {\textcolor{white}{$1$}} ; %
        \node[latent, below left=of 1] (2) {$0$} ; %
        \node[latent, fill=black, below right=of 1] (3) {\textcolor{white}{$1$}} ; %
        \node[latent, fill=black, below left=of 3] (4) {\textcolor{white}{$1$}} ; %
        \factor[below left=of 1] {1-2} {} {} {} ;
        \factor[below right=of 1] {1-3} {} {} {} ;
        \factor[below right=of 2] {2-4} {} {} {} ;
        \factor[below left=of 3] {3-4} {} {} {} ;
        \factoredge[-] {1} {1-2} {2} ; %
        \factoredge[-] {1} {1-3} {3} ; %
        \factoredge[-] {2} {2-4} {4} ; %
        \factoredge[-] {3} {3-4} {4} ; %
        %\edge[-] {2,3} {4} ; %
      } 
    }
    &
    \scalebox{0.5}{
      \tikz{ %
        \node[latent, fill=black] (1) {\textcolor{white}{$1$}} ; %
        \node[latent, fill=black, below left=of 1] (2) {\textcolor{white}{$1$}} ; %
        \node[latent, fill=black, below right=of 1] (3) {\textcolor{white}{$1$}} ; %
        \node[latent, below left=of 3] (4) {$0$} ; %
        \factor[below left=of 1] {1-2} {} {} {} ;
        \factor[below right=of 1] {1-3} {} {} {} ;
        \factor[below right=of 2] {2-4} {} {} {} ;
        \factor[below left=of 3] {3-4} {} {} {} ;
        \factoredge[-] {1} {1-2} {2} ; %
        \factoredge[-] {1} {1-3} {3} ; %
        \factoredge[-] {2} {2-4} {4} ; %
        \factoredge[-] {3} {3-4} {4} ; %
        %\edge[-] {2,3} {4} ; %
      }
    }
    &
    \scalebox{0.5}{
      \tikz{ %
        \node[latent, fill=black] (1) {\textcolor{white}{$1$}} ; %
        \node[latent, fill=black, below left=of 1] (2) {\textcolor{white}{$1$}} ; %
        \node[latent, below right=of 1] (3) {$0$} ; %
        \node[latent, fill=black, below left=of 3] (4) {\textcolor{white}{$1$}} ; %
        \factor[below left=of 1] {1-2} {} {} {} ;
        \factor[below right=of 1] {1-3} {} {} {} ;
        \factor[below right=of 2] {2-4} {} {} {} ;
        \factor[below left=of 3] {3-4} {} {} {} ;
        \factoredge[-] {1} {1-2} {2} ; %
        \factoredge[-] {1} {1-3} {3} ; %
        \factoredge[-] {2} {2-4} {4} ; %
        \factoredge[-] {3} {3-4} {4} ; %
        %\edge[-] {2,3} {4} ; %
      } 
    }
    \\
    \scalebox{0.5}{
      \tikz{ %
        \node[latent] (1) {$0$} ; %
        \node[latent, fill=black, below left=of 1] (2) {\textcolor{white}{$1$}} ; %
        \node[latent, fill=black, below right=of 1] (3) {\textcolor{white}{$1$}} ; %
        \node[latent, below left=of 3] (4) {$0$} ; %
        \factor[below left=of 1] {1-2} {} {} {} ;
        \factor[below right=of 1] {1-3} {} {} {} ;
        \factor[below right=of 2] {2-4} {} {} {} ;
        \factor[below left=of 3] {3-4} {} {} {} ;
        \factoredge[-] {1} {1-2} {2} ; %
        \factoredge[-] {1} {1-3} {3} ; %
        \factoredge[-] {2} {2-4} {4} ; %
        \factoredge[-] {3} {3-4} {4} ; %
        %\edge[-] {2,3} {4} ; %
      }
    }
    &
    \scalebox{0.5}{
      \tikz{ %
        \node[latent, fill=black] (1) {\textcolor{white}{$1$}} ; %
        \node[latent, below left=of 1] (2) {$0$} ; %
        \node[latent, below right=of 1] (3) {$0$} ; %
        \node[latent, fill=black, below left=of 3] (4) {\textcolor{white}{$1$}} ; %
        \factor[below left=of 1] {1-2} {} {} {} ;
        \factor[below right=of 1] {1-3} {} {} {} ;
        \factor[below right=of 2] {2-4} {} {} {} ;
        \factor[below left=of 3] {3-4} {} {} {} ;
        \factoredge[-] {1} {1-2} {2} ; %
        \factoredge[-] {1} {1-3} {3} ; %
        \factoredge[-] {2} {2-4} {4} ; %
        \factoredge[-] {3} {3-4} {4} ; %
        %\edge[-] {2,3} {4} ; %
      } 
    }
    \\
    \scalebox{0.5}{
      \tikz{ %
        \node[latent, fill=black] (1) {\textcolor{white}{$1$}} ; %
        \node[latent, below left=of 1] (2) {$0$} ; %
        \node[latent, below right=of 1] (3) {$0$} ; %
        \node[latent, below left=of 3] (4) {$0$} ; %
        \factor[below left=of 1] {1-2} {} {} {} ;
        \factor[below right=of 1] {1-3} {} {} {} ;
        \factor[below right=of 2] {2-4} {} {} {} ;
        \factor[below left=of 3] {3-4} {} {} {} ;
        \factoredge[-] {1} {1-2} {2} ; %
        \factoredge[-] {1} {1-3} {3} ; %
        \factoredge[-] {2} {2-4} {4} ; %
        \factoredge[-] {3} {3-4} {4} ; %
        %\edge[-] {2,3} {4} ; %
      }
    }
    &
    \scalebox{0.5}{
      \tikz{ %
        \node[latent] (1) {$0$} ; %
        \node[latent, fill=black, below left=of 1] (2) {\textcolor{white}{$1$}} ; %
        \node[latent, below right=of 1] (3) {$0$} ; %
        \node[latent, below left=of 3] (4) {$0$} ; %
        \factor[below left=of 1] {1-2} {} {} {} ;
        \factor[below right=of 1] {1-3} {} {} {} ;
        \factor[below right=of 2] {2-4} {} {} {} ;
        \factor[below left=of 3] {3-4} {} {} {} ;
        \factoredge[-] {1} {1-2} {2} ; %
        \factoredge[-] {1} {1-3} {3} ; %
        \factoredge[-] {2} {2-4} {4} ; %
        \factoredge[-] {3} {3-4} {4} ; %
        %\edge[-] {2,3} {4} ; %
      } 
    }
    &
    \scalebox{0.5}{
      \tikz{ %
        \node[latent] (1) {$0$} ; %
        \node[latent, below left=of 1] (2) {$0$} ; %
        \node[latent, below right=of 1] (3) {$0$} ; %
        \node[latent, fill=black, below left=of 3] (4) {\textcolor{white}{$1$}} ; %
        \factor[below left=of 1] {1-2} {} {} {} ;
        \factor[below right=of 1] {1-3} {} {} {} ;
        \factor[below right=of 2] {2-4} {} {} {} ;
        \factor[below left=of 3] {3-4} {} {} {} ;
        \factoredge[-] {1} {1-2} {2} ; %
        \factoredge[-] {1} {1-3} {3} ; %
        \factoredge[-] {2} {2-4} {4} ; %
        \factoredge[-] {3} {3-4} {4} ; %
        %\edge[-] {2,3} {4} ; %
      } 
    }
    &
    \scalebox{0.5}{
      \tikz{ %
        \node[latent] (1) {$0$} ; %
        \node[latent, below left=of 1] (2) {$0$} ; %
        \node[latent, fill=black, below right=of 1] (3) {\textcolor{white}{$1$}} ; %
        \node[latent, below left=of 3] (4) {$0$} ; %
        \factor[below left=of 1] {1-2} {} {} {} ;
        \factor[below right=of 1] {1-3} {} {} {} ;
        \factor[below right=of 2] {2-4} {} {} {} ;
        \factor[below left=of 3] {3-4} {} {} {} ;
        \factoredge[-] {1} {1-2} {2} ; %
        \factoredge[-] {1} {1-3} {3} ; %
        \factoredge[-] {2} {2-4} {4} ; %
        \factoredge[-] {3} {3-4} {4} ; %
        %\edge[-] {2,3} {4} ; %
      } 
    }
    \\
    % All Zeros %
    \scalebox{0.5}{
      \tikz{ %
        \node[latent] (1) {$0$} ; %
        \node[latent, below left=of 1] (2) {$0$} ; %
        \node[latent, below right=of 1] (3) {$0$} ; %
        \node[latent, below left=of 3] (4) {$0$} ; %
        \factor[below left=of 1] {1-2} {} {} {} ;
        \factor[below right=of 1] {1-3} {} {} {} ;
        \factor[below right=of 2] {2-4} {} {} {} ;
        \factor[below left=of 3] {3-4} {} {} {} ;
        \factoredge[-] {1} {1-2} {2} ; %
        \factoredge[-] {1} {1-3} {3} ; %
        \factoredge[-] {2} {2-4} {4} ; %
        \factoredge[-] {3} {3-4} {4} ; %
        %\edge[-] {2,3} {4} ; %
      } 
    }
    \\
    \end{tabular}
  \\ \hline
  % Mismatched %
  $\operatorname{E}(\mathbf{y}) = 2$
  &
  \begin{tabular}{cccc}
    \scalebox{0.5}{
      \tikz{ %
        \node[latent] (1) {$0$} ; %
        \node[latent, fill=black, below left=of 1] (2) {\textcolor{white}{$1$}} ; %
        \node[latent, below right=of 1] (3) {$0$} ; %
        \node[latent, fill=black, below left=of 3] (4) {\textcolor{white}{$1$}} ; %
        \factor[below left=of 1] {1-2} {} {} {} ;
        \factor[below right=of 1] {1-3} {} {} {} ;
        \factor[below right=of 2] {2-4} {} {} {} ;
        \factor[below left=of 3] {3-4} {} {} {} ;
        \factoredge[-] {1} {1-2} {2} ; %
        \factoredge[-] {1} {1-3} {3} ; %
        \factoredge[-] {2} {2-4} {4} ; %
        \factoredge[-] {3} {3-4} {4} ; %
        %\edge[-] {2,3} {4} ; %
      }
    }
    &
    \scalebox{0.5}{
      \tikz{ %
        \node[latent, fill=black] (1) {\textcolor{white}{$1$}} ; %
        \node[latent, below left=of 1] (2) {$0$} ; %
        \node[latent, fill=black, below right=of 1] (3) {\textcolor{white}{$1$}} ; %
        \node[latent, below left=of 3] (4) {$0$} ; %
        \factor[below left=of 1] {1-2} {} {} {} ;
        \factor[below right=of 1] {1-3} {} {} {} ;
        \factor[below right=of 2] {2-4} {} {} {} ;
        \factor[below left=of 3] {3-4} {} {} {} ;
        \factoredge[-] {1} {1-2} {2} ; %
        \factoredge[-] {1} {1-3} {3} ; %
        \factoredge[-] {2} {2-4} {4} ; %
        \factoredge[-] {3} {3-4} {4} ; %
        %\edge[-] {2,3} {4} ; %
      }
    }
    \\
  \end{tabular}
\end{tabular}


in oe said above,   e want to encode 


For each block, we will assign it one of KTo each block, we will assign it to some neighborhood


We we don't know exactly what makes blocks similar or
different. However, for each block we do know about a number of
attributes that we can associate with the block. We can start with
attributes that have proven to be sociological salient like such as
number of people who live on the block, their racial and ethnic
composition, and distribution of age of the housing stock.

Between blocks, we can define a variety of similarity functions for
these attributes, like the absolute difference in the percentage of 
white people, or the Jensen-Shannon divergence in the age of the
housing stock. We can assign these measures to the edges between blocks.

We also may know some things about the edges themselves

To the edges between blocks, we can assign measures of the similarity
of the these block features. 

[[new figure here]]




This representation of the city is also a representation of a Markov
Random Field.



To the edges between blocks, we can associate 

 and can assign our blocks 


\section*{Mathematics}

\section*{Results}

\section*{Conclusion}

\end{document}
