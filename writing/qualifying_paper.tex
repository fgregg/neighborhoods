\documentclass[12pt,letter]{article}\usepackage[]{graphicx}\usepackage[]{color}
%% maxwidth is the original width if it is less than linewidth
%% otherwise use linewidth (to make sure the graphics do not exceed the margin)
\makeatletter
\def\maxwidth{ %
  \ifdim\Gin@nat@width>\linewidth
    \linewidth
  \else
    \Gin@nat@width
  \fi
}
\makeatother

\definecolor{fgcolor}{rgb}{0.345, 0.345, 0.345}
\newcommand{\hlnum}[1]{\textcolor[rgb]{0.686,0.059,0.569}{#1}}%
\newcommand{\hlstr}[1]{\textcolor[rgb]{0.192,0.494,0.8}{#1}}%
\newcommand{\hlcom}[1]{\textcolor[rgb]{0.678,0.584,0.686}{\textit{#1}}}%
\newcommand{\hlopt}[1]{\textcolor[rgb]{0,0,0}{#1}}%
\newcommand{\hlstd}[1]{\textcolor[rgb]{0.345,0.345,0.345}{#1}}%
\newcommand{\hlkwa}[1]{\textcolor[rgb]{0.161,0.373,0.58}{\textbf{#1}}}%
\newcommand{\hlkwb}[1]{\textcolor[rgb]{0.69,0.353,0.396}{#1}}%
\newcommand{\hlkwc}[1]{\textcolor[rgb]{0.333,0.667,0.333}{#1}}%
\newcommand{\hlkwd}[1]{\textcolor[rgb]{0.737,0.353,0.396}{\textbf{#1}}}%

\usepackage{framed}
\makeatletter
\newenvironment{kframe}{%
 \def\at@end@of@kframe{}%
 \ifinner\ifhmode%
  \def\at@end@of@kframe{\end{minipage}}%
  \begin{minipage}{\columnwidth}%
 \fi\fi%
 \def\FrameCommand##1{\hskip\@totalleftmargin \hskip-\fboxsep
 \colorbox{shadecolor}{##1}\hskip-\fboxsep
     % There is no \\@totalrightmargin, so:
     \hskip-\linewidth \hskip-\@totalleftmargin \hskip\columnwidth}%
 \MakeFramed {\advance\hsize-\width
   \@totalleftmargin\z@ \linewidth\hsize
   \@setminipage}}%
 {\par\unskip\endMakeFramed%
 \at@end@of@kframe}
\makeatother

\definecolor{shadecolor}{rgb}{.97, .97, .97}
\definecolor{messagecolor}{rgb}{0, 0, 0}
\definecolor{warningcolor}{rgb}{1, 0, 1}
\definecolor{errorcolor}{rgb}{1, 0, 0}
\newenvironment{knitrout}{}{} % an empty environment to be redefined in TeX

\usepackage{alltt}
\usepackage{amsmath}
\usepackage{url}
\usepackage{tikz}
\usepackage{adjustbox}

\usetikzlibrary{arrows}
\usetikzlibrary{bayesnet}

\DeclareMathOperator*{\argmin}{\arg\!\min}
\DeclareMathOperator*{\argmax}{\arg\!\max}
\IfFileExists{upquote.sty}{\usepackage{upquote}}{}

\begin{document}










\section*{Introduction}
Social scientists have long been concerned in dividing the social
world into meaningful components. In recent year's we have
developed many methods to accomplish this when know what parts of the
world are similar to one another, broadly clustering, community
detection, and network partition techniques. Unfortunately, we are
seldom so lucky as to already know what really matters for making one
thing similar to another. 

When our similarity is binary, i.e. did ego nominate alter as friend,
life is easy. But, as soon as a variable has more than two possible
values then the researcher must choose, by hand, how to turn those
values into similarities. In the end, we adjust the various weights
until the groupings from our favorite clustering algorithm come out
looking about right. We would have been better off just drawing what
we wanted in the first place.

Another way is possible. If our subjects ever group themselves, we can
use that revealed grouping to learn what counts for similarity in a particular
social world. We can then apply our derived similarity to group elements in
a way that is likely to be meaningful to the types of people we learned from.

People group themselves in a variety of ways: making friends, getting
married, joining organizations, share consumption, and
self-labeling. In this paper, we will discuss a particular form of
self-labeling: claims that a location belongs to a city
neighborhood. However, the larger approach is much more general.

\section*{Neighborhood Model Representation}
Let's imagine a very small city consisting of four blocks. We can
represent this tiny town as a network where two blocks are connected
if they face the same street. In this representation, blocks that are
kitty corner are not directly connected. We'll index the blocks as
$1$, $2$, $3$, and $4$.

\begin{figure}[h]
\centering
\tikz{
\draw[help lines] (0,0) grid (2,2);
\node at (0.5, 0.5) {3} ;
\node at (1.5, 1.5) {2} ;
\node at (0.5, 1.5) {1} ;
\node at (1.5, 0.5) {4} ;
}
\end{figure}

\begin{figure}[h]
\centering

\tikz{ %
  \node[latent] (1) {$1$} ; %
  \node[latent, below left=of 1] (2) {$2$} ; %
  \node[latent, below right=of 1] (3) {$3$} ; %
  \node[latent, below left=of 3] (4) {$4$} ; %
  \edge[-] {2,3} {1} ; %
  \edge[-] {2,3} {4} ; %
}

\end{figure}

Suppose that, in our city, there are two neighborhoods. Each block
belongs to either one or the other of these neighborhoods. Neighboring
blocks that are similar are more apt to belong to the same
neighborhood and neighboring blocks that are different are more apt to
belong to different neighborhoods.

We want similar, neighboring blocks to belong to the same
neighborhood. One way to formalize this desire is to score every
possible pattern of neighborhood labels in such a way that our
preferred patterns have the best score.

First, we need some more precise terms. Let the two neighborhoods be
called $0$ and $1$. Every block belongs to either neighborhood $0$ or
$1$. We will denote this membership as $y_i$, so that $y_1=0$ is
equivalent to saying that block $1$ belongs to the $0$
neighborhood. Let the similarity between blocks $i$ and $j$ be called
$\phi_{i,j}$.

\begin{figure}[!h]
\centering

\tikz{ %
  \node[latent] (1) {$y_1$} ; %
  \node[latent, below left=of 1] (2) {$y_2$} ; %
  \node[latent, below right=of 1] (3) {$y_3$} ; %
  \node[latent, below left=of 3] (4) {$y_4$} ; %
  \factor[below left=of 1] {1-2} {$\phi_{1,2}$} {} {} ;
  \factor[below right=of 1] {1-3} {$\phi_{1,3}$} {} {} ;
  \factor[below right=of 2] {2-4} {$\phi_{2,4}$} {} {} ;
  \factor[below left=of 3] {3-4} {$\phi_{3,4}$} {} {} ;
  \factoredge[-] {1} {1-2} {2} ; %
  \factoredge[-] {1} {1-3} {3} ; %
  \factoredge[-] {2} {2-4} {4} ; %
  \factoredge[-] {3} {3-4} {4} ; %
  %\edge[-] {2,3} {4} ; %
}

\end{figure}


Let a particular pattern of assignment of blocks to neighborhoods
be called $\mathbf{y}$.  The score of $\mathbf{y}$ will be
$\operatorname{E}(\mathbf{y})$\footnote{$\operatorname{E}$ as in
  energy not expectation} which will take the following form:

\begin{align}
\operatorname{E}(\mathbf{y}) = \sum_{<i,j>}^{\mathcal{N}}\epsilon_{i,j}(y_i,y_j,\phi_{i,j})
\end{align}

Where $<i,j>$ indexes a pair of neighboring blocks, and where $i < j$.
$\mathcal{N}$ are all these indices of neighboring blocks.

\begin{equation}
\epsilon_{i,j}(y_i,y_j,\phi_{i,j}) = \begin{cases}
  0 &y_i = y_j \\
  \phi_{i,j} &y_i \neq y_j
\end{cases}
\end{equation}

Suppose that we want our preferred neighborhood assignment to have the
lowest score. If our similarity measures $\phi_{i,j}$ are positive
when blocks are similar and negative when blocks are different, then
we will encourage similar, neighboring blocks to belong to the same
neighborhood. 

Let our city have the following $\phi$'s.

\begin{align*}
&\phi_{1,2} = 1 \\
&\phi_{1,3} = -1 \\
&\phi_{2,4} = -1 \\
&\phi_{3,4} = 1
\end{align*} 

\noindent
If we now score every possible pattern of neighborhood labels, we will
find that the lowest scoring assignments are ones that put blocks $1$
and $2$ in one neighborhood and $3$ and $4$ in the other (Table
\ref{table:lowest}, Table \ref{table:energy}). By choosing the right
\phi$'s we can have made our preferred pattern have the lowest scores.

\begin{table}
\centering
  \begin{tabular}{cc}
      \tikz{ %
        \node[latent] (1) {$y_1$} ; %
        \node[latent, below left=of 1] (2) {$y_2$} ; %
        \node[latent, fill=black, below right=of 1] (3) {\textcolor{white}{$y_3$}} ; %
        \node[latent, fill=black, below left=of 3] (4) {\textcolor{white}{$y_4$}} ; %
        \factor[below left=of 1] {1-2} {$1$} {} {} ;
        \factor[below right=of 1] {1-3} {$-1$} {} {} ;
        \factor[below right=of 2] {2-4} {$-1$} {} {} ;
        \factor[below left=of 3] {3-4} {$1$} {} {} ;
        \factoredge[-] {1} {1-2} {2} ; %
        \factoredge[-] {1} {1-3} {3} ; %
        \factoredge[-] {2} {2-4} {4} ; %
        \factoredge[-] {3} {3-4} {4} ; %
      } 
    &
      \tikz{ %
        \node[latent, fill=black] (1) {\textcolor{white}{$y_1$}} ; %
        \node[latent, fill=black, below left=of 1] (2) {\textcolor{white}{$y_2$}} ; %
        \node[latent, below right=of 1] (3) {$y_3$} ; %
        \node[latent, below left=of 3] (4) {$y_4$} ; %
        \factor[below left=of 1] {1-2} {$1$} {} {} ;
        \factor[below right=of 1] {1-3} {$-1$} {} {} ;
        \factor[below right=of 2] {2-4} {$-1$} {} {} ;
        \factor[below left=of 3] {3-4} {$1$} {} {} ;
        \factoredge[-] {1} {1-2} {2} ; %
        \factoredge[-] {1} {1-3} {3} ; %
        \factoredge[-] {2} {2-4} {4} ; %
        \factoredge[-] {3} {3-4} {4} ; %
      } 
    \\
  \end{tabular}
  \caption{Preferred Assignment: If $y_i = 0$, the block is colored
    white. If $y_i = 1$, the block is black.}
  \label{table:lowest}
\end{table}

\begin{table}[h]
\input{energy_table.tex}
\caption{Scores of Neighborhood Assignments}
\label{table:energy}
\end{table}

For small networks, can check the scores for each individual pattern
to see if our preferred pattern really has the lowest score. This
exhaustive strategy becomes quickly impractical for larger
cities. With two possible neighborhoods, the number of possible
assignments is $2^N$ where N is the number of blocks. This
permutational explosion means that for even small cities, we cannot
possibly check every possible neighborhood assignment in human-scale
time. 

\subsection*{Network Methods}
Fortunately, researchers, largely in the field of computer vision,
have developed methods to quickly find the lowest scoring assignment
for problems like ours. This work traces back to a 1986 paper by
Greig, Porteous, and Sehult, where they demonstrated that finding
lowest scoring assignment for a two neighborhood case was equivalent
to solving the problem of finding the minimum cut of a
graph.\cite{greig_exact_1989}

In that paper, the authors set up an scoring problem that is slightly
different than ours. In addition to terms that depended upon pairs of
neighbors, their score also included terms for individual nodes.

\begin{align}
\operatorname{E}(\mathbf{y}) = \sum_i\epsilon_i(y_i) + \sum_{<i
  j>}^{\mathcal{N}}\epsilon_{i,j}(y_i,y_j,\phi_{i,j})
\end{align}

\noindent
The node terms were of one of two forms, either

\begin{equation}
\epsilon_{i}(y_i) = \begin{cases}
  0 &y_i = 0 \\
  \phi_{i} \geq 0 &y_i = 1
\end{cases}
\end{equation}

\noindent
or 

\begin{equation}
\epsilon_{i}(y_i) = \begin{cases}
  \phi_{i} \geq 0 &y_i = 0 \\
  0 &y_i = 1
\end{cases}
\end{equation}

\noindent
They had the same form of $\epsilon_{i,j}$

\begin{equation}
\epsilon_{i,j}(y_i,y_j,\phi_{i,j}) = \begin{cases}
  0 &y_i = y_j \\
  \phi_{i,j} &y_i \neq y_j
\end{cases}
\end{equation}

\noindent
but unlike us, $\phi_{i,j} \geq 0$ for all $<i, j>$ in $\mathcal{N}$. 

One of their networks could look like this

\begin{figure}[!h]
\centering

\tikz{ %
  \node[latent] (1) {$y_1$} ; %
  \node[latent, below left=of 1] (2) {$y_2$} ; %
  \node[latent, below right=of 1] (3) {$y_3$} ; %
  \node[latent, below left=of 3] (4) {$y_4$} ; %
  \factor[below left=of 1] {1-2} {$1$} {} {} ;
  \factor[below right=of 1] {1-3} {$1$} {} {} ;
  \factor[below right=of 2] {2-4} {$1$} {} {} ;
  \factor[below left=of 3] {3-4} {$1$} {} {} ;
  \factoredge[-] {1} {1-2} {2} ; %
  \factoredge[-] {1} {1-3} {3} ; %
  \factoredge[-] {2} {2-4} {4} ; %
  \factoredge[-] {3} {3-4} {4} ; %
  %\edge[-] {2,3} {4} ; %
}

\end{figure}

Based on this undirected network, Greig and his co-authors constructed
a special, directed network where every edge has a cost associated
with it. For every variable $y_i$ in the undirected network, the
directed network contains a node $z_i$.  For every edge $(y_i, y_j)$ in the
undirected graph, they introduced a directed edge from $z_i$ to $z_j$ and
another from $z_j$ to $z_i$ with associated costs of $\phi_{i,j}$.

In addition to the original nodes, they also introduced two special
nodes: a source node called $s$ and a target, or sink, node called
$t$. If $\epsilon_i(0) = 0$ then there would be a directed edge from
$s$ to the $i$th node. If $\epsilon_i(1) = 0$ then there would be
directed edge from the $i$th node to $t$. Networks like this are
called s-t networks.

Suppose that $\epsilon_1(1)=2$, $\epsilon_2(0)=1$, $\epsilon_3(1)=3$,
and $\epsilon_4(0)=3$. The corresponding directed graph is shown in
Figure \ref{fig:directed}. 

\begin{figure}[!h]
\centering

\begin{tikzpicture}[auto, >= stealth', shorten >= 1pt, node
    distance=2cm, thick]
\tikzset{vertex/.style = {shape=circle, draw, minimum size = 1.5em}}
\tikzset{Dedge/.style = {->}}
\tikzset{Uedge/.style = {<->}}

\node[vertex, above=of 1] (0) {$s$} ; %
\node[vertex] (1) {$y_1$} ; %
\node[vertex, below left=of 1] (2) {$y_2$} ; %
\node[vertex, below right=of 1] (3) {$y_3$} ; %
\node[vertex, below left=of 3] (4) {$y_4$} ; %
\node[vertex, below=of 4] (5) {$t$} ; %
\path
  (0) edge [Dedge] node {2} (1)
  (0) edge [Dedge, bend left] node[right] {3} (3)
  (1) edge [Uedge] node {1} (2) 
  (1) edge [Uedge] node {1} (3) 
  (2) edge [Uedge] node {1} (4) 
  (3) edge [Uedge] node {1} (4) 
  (2) edge [Dedge, bend right] node[left] {1} (5)
  (4) edge [Dedge] node {3} (5)  ;

\end{tikzpicture}
\caption{S-T Network}
\label{fig:directed}
\end{figure}

We must now introduce some terms. First, to `cut' a network is to
remove edges from the network so that the network is split into two
unconnected, smaller networks. Second, an `s-t cut' is a cut on a s-t
network that causes the source node and target node to end up in
separate, unconnected networks. Finally, a `minimum cut' is an s-t cut
where the sum of cost associated with the removed edges is as small as
any other possible s-t cut. 

In our example, removing the edges $(z_2, t)$ and $(z_4, t)$ is an s-t
cut, but not the minimum cut. The minimum cut would be removing edges
$(z_1, z_2)$ and $(z_3, z_4)$ (Figure \ref{fig:mincut}).

\begin{figure}[!h]
\centering

\begin{tikzpicture}[auto, >= stealth', shorten >= 1pt, node
    distance=2cm, thick]
\tikzset{vertex/.style = {shape=circle, draw, minimum size = 1.5em}}
\tikzset{Dedge/.style = {->}}
\tikzset{Uedge/.style = {<->}}

\node[vertex, above=of 1] (0) {$s$} ; %
\node[vertex] (1) {$y_1$} ; %
\node[vertex, below left=of 1] (2) {$y_2$} ; %
\node[vertex, below right=of 1] (3) {$y_3$} ; %
\node[vertex, below left=of 3] (4) {$y_4$} ; %
\node[vertex, below=of 4] (5) {$t$} ; %
\path
  (0) edge [Dedge] node {2} (1)
  (0) edge [Dedge, bend left] node[right] {3} (3)
  (1) edge [Uedge] node {1} (3) 
  (2) edge [Uedge] node {1} (4) 
  (2) edge [Dedge, bend right] node[left] {1} (5)
  (4) edge [Dedge] node {3} (5)  ;

\end{tikzpicture}
\caption{Minimum Cut}
\label{fig:mincut}
\end{figure}

Every s-t cut can be mapped onto a pattern of neighborhood block
labels by letting $y_i = 1$ if node $z_i$ ends up in the $t$
subnetwork and $y_i = 0$ otherwise. Greig and his coauthors
demonstrated that the minimum s-t cut their constructed s-t network
always corresponded to the minimum scoring pattern of block
assignment.

Because computer scientists have known how to solve the minimum cut
problem swiftly since the 1950s, this meant that a large class of
network configurations problems were now tractable. Instead of
checking every possible pattern out of exponentially possibilities, we
can now find the best scoring pattern directly and
quickly.\cite{ford_maximal_1956}

\subsubsection*{Submodularity}
Greig's original result only applied to scoring functions 
where blocks could belong to only one of two classes and where the
scoring function had an important property called submodularity.  A
submodular scoring function would be one where $\epsilon_{i,j}(0,0) +
\epsilon_{i,j}(1,1) \leq \epsilon_{i,j}(0,1) + \epsilon_{i,j}(1,0)$
for all $i,j$ in $\mathcal{N}$. Our scoring function is decisively not
submodular because we want it to cost more to put together very
dissimilar neighboring blocks than to assign them to them to different
neighborhoods. That is, for very dissimilar blocks we want it to hold
that $\epsilon_{i,j}(0,0) + \epsilon_{i,j}(1,1) \boldsymbol{>}
\epsilon_{i,j}(0,1) + \epsilon_{i,j}(1,0)$.

Since we don't have a submodular scoring functions and often will have
more than two neighborhood names, we can't use the minimum cuts
algorithm directly to find the lowest scoring neighborhood
pattern. However, since the 1980s computers scientists have developed
other graph cutting methods that allow us to find approximate
solutions to multi-label non-submodular scoring problems like ours
\cite{something}.

\section*{Learning similarities}
In the previous section, we showed how we could construct a scoring
function so that a preferred pattern of block assignments would have
the best score. Now we will turn to how, given a preferred assignment,
we can learn such a scoring function. 

We will first make our lives easier by only considering simple, linear
similarity terms. As before, the scoring function will be

\begin{align}
\operatorname{E}(\mathbf{y}) = \sum_{<i,j>}^{\mathcal{N}}\epsilon_{i,j}(y_i,y_j,\phi_{i,j})
\end{align}

\noindent
and

\begin{equation}
\epsilon_{i,j}(y_i,y_j,\phi_{i,j}) = \begin{cases}
  0 &y_i = y_j \\
  \phi_{i,j} &y_i \neq y_j
\end{cases}
\end{equation}

\noindent
but now, let the similarity term $\phi_{i,j}$ be the weighted sum of
observed similarities between block $i$ and $j$. 

\begin{align}
\phi_{i,j} = w_0 + w_1s_{1,i,j} + w_2s_{2,i,j} + ... + w_ns_{n,i,j}
\end{align} 

\noindent
An observed inter-block similarity could be absolute difference in
population, or dummy variable indicating whether the blocks are
separated by a railroad, or similar.

\noindent

With this form, learning a scoring function means finding weights
that give a better score to our preferred pattern of neighborhood
assignments than any other pattern. In other words, we want to find
some weights, $\mathbf{w}$, that solves this system of linear
equations:

\begin*{equations}
\operatorname{E}(\mathbf{y}_1, \mathbf{s}, \mathbf{w})
\geq \operatorname{E}(\mathbf{y}^*, \mathbf{s}, \mathbf{w}) \\
\operatorname{E}(\mathbf{y}_2, \mathbf{s}, \mathbf{w})
\geq \operatorname{E}(\mathbf{y}^*, \mathbf{s}, \mathbf{w})
\ldots \\
\operatorname{E}(\mathbf{y}_{M-1}, \mathbf{s}, \mathbf{w})
\geq \operatorname{E}(\mathbf{y}^*, \mathbf{s}, \mathbf{w})
\operatorname{E}(\mathbf{y}_{M}, \mathbf{s}, \mathbf{w})
\geq \operatorname{E}(\mathbf{y}^*, \mathbf{s}, \mathbf{w})
\end{equations}

Where \mathbf{y}^* is our preferred pattern, and M is the number of
possible neighborhood patterns.

\subsubsection*{Specifying a unique solution}
In order to use some convenient machinery, we will modify the learning
problem somewhat. These modifications will only act to specify a
particular solution to the system of inequalities. Any solution to the
modified problems is also a solution to the system of inequalities and
if there is solution to the system of inequalities there will be a
solution to the modified problems.

First, observe that if a set of weights gives the preferred assignment
a lower score than any other assignment, there is some difference
between the score of the preferred assignment and the score of the
next best scoring assignment. Call this difference the 'margin. Our
first modification to the learning problem is that we now seek the
weights that give our preferred assignment the largest margin.

\begin{align}
&\argmax_{\mathbf{w}} \mathbf{\gamma} \\
&\text{such that} \\
&\operatorname{E}(\mathbf{y}, \mathbf{s}, \mathbf{w})
- \operatorname{E}(\mathbf{y}^*, \mathbf{s}, \mathbf{w}) \geq \gamma\\ 
&\text{for all } \mathbf{y} \text{ where } \mathbf{y} \text{ is in the set of
  possible neighborhood assignments}\\
&\text{and } \mathbf{y} \neq \mathbf{y}^*
\end{align}

This still does not specify a unique set of weights. To do that, we
constrain the sizes of the weights. We could do this by
limiting the sum of the absolute values of the weights. A similar, but
more mathematically convenient choice is to require that
$\sqrt{\sum_i^M w_i^2 = 1}$. 

%
\begin{align*}
&\argmax_{\mathbf{w}:\sqrt{\sum_i^M w_i^2 = 1}=1} \mathbf{\gamma} \\
&\text{such that} \\
&\operatorname{E}(\mathbf{y}, \mathbf{s}, \mathbf{w})
- \operatorname{E}(\mathbf{y}^*, \mathbf{s}, \mathbf{w}) \geq \gamma\\ 
&\text{for all } \mathbf{y} \text{ where } \mathbf{y} \text{ is in the set of
  possible neighborhood assignments}\\
&\text{and } \mathbf{y} \neq \mathbf{y}^*
\end{align*}
%

Here $\mathbf{y}*$ is the target neighborhood assignment and $\mathbf{s}$
are similarity measures between blocks. 

Large margin problems like this one have an equivalent canonical
representation: 
%
\begin{align*}
&\argmin_{\mathbf{w}} \frac{1}{2}||\mathbf{w}||^2 \\
&\text{such that} \\
&\operatorname{E}(\mathbf{y}, \mathbf{s}, \mathbf{w})
- \operatorname{E}(\mathbf{y}^*, \mathbf{s}, \mathbf{w}) \geq 1 \\ 
&\text{for all } \mathbf{y} \text{ where } \mathbf{y} \text{ is in the set of
  possible neighborhood assignments}\\
&\text{and } \mathbf{y} \neq \mathbf{y}^*
\end{align*}

This is quadratic program, a well known class of constrained
optimization problems. Unfortunately, we cannot solve this problem
directly because the number of constraints is typically too
large. That is, we require every possible pattern to have a
higher score than our target pattern, and the number of possible
patterns grows exponentially with the number of blocks.

Instead, we'll solve similar problem that, in practice, we can solve
quickly, and will also learn a set of weights that still will give our
target pattern lower energy than any other
pattern.\cite{szummer_learning_2008}
 

\subsection*{Learning Routine}
First, we initialize the weights to some starting value, create an
empty set of constraints $\mathcal{K}$, and set a counter $i$ to $0$.
\begin{enumerate}
\item Use a minimum cuts algorithm to find the neighborhood pattern
  that has a lower score than any pattern given the current weights.
  Call this pattern $\mathbf{y}_i$ and add it to the constraint
  set $\mathcal{K}$.

\item Update the weights by solving the quadratic program: 
%
\begin{align*}
&\argmin_{\mathbf{w}} \frac{1}{2}||\mathbf{w}||^2 \\
&\text{such that} \\
&\operatorname{E}(\mathbf{y}, \mathbf{s}, \mathbf{w})
- \operatorname{E}(\mathbf{y}^*, \mathbf{s}, \mathbf{w}) \geq 1 \\ 
&\text{for all } \mathbf{y} \text{ where } \mathbf{y} \text{ is in } \mathcal{K}\\
&\text{and } \mathbf{y} \neq \mathbf{y}^*
\end{align*}
%

\item If the weights changed in the previous step, set $i = i + 1$ and
  go to step 1. If the weights did not change, stop the routine.
\end{enumerate}

If we find a set of weights that gives our target pattern a lower
energy then any pattern in the constraint set $\mathcal{K}$, then we
are assured that our target pattern has a lower energy than any other
possible pattern. If another pattern had a lower score given the final
set of weights, then it would have been added to $\mathcal{K}$ in step
1, and our optimization routine would have continued. 

\subsection*{Empirical Loss}
Some readers may have noticed that the our original four-city block example
still does not have a unique solution for the learning problem we have
set up. There are two different neighborhood assignments that have the
same lowest score, and they both achieve our goal of putting similar
blocks in the same neighborhood and dissimilar blocks in different
neighborhoods (Table \ref{table:lowest}).

\begin{table}
\centering
  \begin{tabular}{cc}
      \tikz{ %
        \node[latent] (1) {$0$} ; %
        \node[latent, below left=of 1] (2) {$0$} ; %
        \node[latent, fill=black, below right=of 1] (3) {\textcolor{white}{$1$}} ; %
        \node[latent, fill=black, below left=of 3] (4) {\textcolor{white}{$1$}} ; %
        \factor[below left=of 1] {1-2} {} {} {} ;
        \factor[below right=of 1] {1-3} {} {} {} ;
        \factor[below right=of 2] {2-4} {} {} {} ;
        \factor[below left=of 3] {3-4} {} {} {} ;
        \factoredge[-] {1} {1-2} {2} ; %
        \factoredge[-] {1} {1-3} {3} ; %
        \factoredge[-] {2} {2-4} {4} ; %
        \factoredge[-] {3} {3-4} {4} ; %
      } 
    &
      \tikz{ %
        \node[latent, fill=black] (1) {\textcolor{white}{$1$}} ; %
        \node[latent, fill=black, below left=of 1] (2) {\textcolor{white}{$1$}} ; %
        \node[latent, below right=of 1] (3) {$0$} ; %
        \node[latent, below left=of 3] (4) {$0$} ; %
        \factor[below left=of 1] {1-2} {} {} {} ;
        \factor[below right=of 1] {1-3} {} {} {} ;
        \factor[below right=of 2] {2-4} {} {} {} ;
        \factor[below left=of 3] {3-4} {} {} {} ;
        \factoredge[-] {1} {1-2} {2} ; %
        \factoredge[-] {1} {1-3} {3} ; %
        \factoredge[-] {2} {2-4} {4} ; %
        \factoredge[-] {3} {3-4} {4} ; %
      } 
    \\
  \end{tabular}
  \caption{Equivalent Lowest Assignments}
  \label{table:lowest}
\end{table}

Happily, and for unrelated reasons, practicioners have found that good
performance for this type of learning problem depends upon incorporating an
empirical loss function into the objective function. If we follow
suit, we will also side step this problem of multiple equivalent
assignments. 

The learning routine is largely the same as above, but we change the
quadratic program in step 2:
%
\begin{align*}
&\argmin_{\mathbf{w}} \frac{1}{2}||\mathbf{w}||^2 +
  C\cdot\xi\\
&\text{such that} \\
&\operatorname{E}(\mathbf{y}, \mathbf{s}, \mathbf{w})
- \operatorname{E}(\mathbf{y}^*, \mathbf{s}, \mathbf{w}) \geq
\Delta(\mathbf{y}^*, \mathbf{y}) - \xi\\ 
& \xi \geq 0\\
&\text{for all } \mathbf{y} \text{ where } \mathbf{y} \text{ is in the set of
  possible neighborhood assignments,}\\
&\xi \text{ is a slack variable, and } \Delta \text{ is an empirical loss.}\\
\end{align*}
%

The key to solving this problem is finding assignments that have the
lowest score, where the score incorporates the empirical loss: i.e.
%
\begin{align}
\operatorname{E}(\mathbf{y}) = \sum_{<i j>}^{\mathcal{N}}\epsilon_{i,j}(y_i,y_j) + C\cdot\Delta(\mathbf{y}*,
\mathbf{y})
\end{align}
%
If the loss can be decomposed over the blocks like
%
\begin{align}
\operatorname{E}(\mathbf{y}) = \sum_{<i j>}^{\mathcal{N}}\epsilon_{i,j}(y_i,y_j) + C\cdot\sum_i^N\Delta(y_i*, y_i)
\end{align}
%
then the objective has the form that allows us to use graph cutting
problems to find an exact or approximate minimum pattern. If the
loss function does not decompose we have to check an exponentially
many possible assignment, typically an intractable problem.

Ideally, we want loss function that increases the further we are from our
target assignment. A natural choice is to penalize every block that is
incorrectly assigned. If we penalize every error the same then we have
the classic Hamming loss. 

$\sum_i^N\begin{cases}
  0 &y_i^* = y_i \\
  1 &y_i^* \neq y_i
\end{cases}$. 

Incorporating this empirical loss into our learning problem means
that, of a set of previously equivalent assignments, only the
assignment closest to the training data will be have the lowest score.


For the same reason, traditional Markov Chain Monte Carlo
methods for finding the lowest scoring assignment also prove
intractable.\footnote{In order to converge on the mode of the
  distribution of scores, we have to calculate a normalizing constant,
  which is the sum of the scores of all possible assignments. This is
  computationally too expensive. There have been a number of attempts
  to find an acceptable substitute for the normalizing constant, but
  the empirical results have disappointed.\cite{li_mrf_2009}}


\section*{Results}
So much for theory, let's see how this works in practice. Let's take
the case of Chicago neighborhoods.  

\subsection*{Data}




I have a nightly updated database of geocoded Craigslist apartment
rentals, sublet, and roommate listings. For most of these listings,
the poster entered some text in a ``Specific Location'' field. With
some minimal pre-processing, we can use these data as observations of
claims that geographical points are in some neighborhood.

Using kernel density estimation, we can use this point data to
estimate a continuous probability distributions that any point in the
city will be claimed to be in any of the neighborhoods. 

At the center of every census block, we will calculate the most
probable neighborhood assignment and assign that most likely
neighborhood to that block. This will be our training data (Figure
\ref{fig:training}).

I would like to treat this measure as a kind of `averaged
neighborhood perception.' Of course, it is at best a biased measure of
that perception. There are enormous selection effect, but perhaps more
troublesome is that listers are likely to claim neighborhoods
strategically in order to increase the desirability of their
listing. I have been looking for a scientific samples of neighborhood
perception that I can compare against so as to get a sense of the
magnitude of the biases. 

However, I think it is possible that these data, biases and all, may
be good enough. We'll have ways of checking that hope.


\begin{figure}
\begin{knitrout}
\definecolor{shadecolor}{rgb}{0.969, 0.969, 0.969}\color{fgcolor}

{\centering \includegraphics[width=\maxwidth]{/home/fgregg/sweave-cache/figs/plotTraining} 

}



\end{knitrout}

\caption{Training Data}
\label{fig:training}
\end{figure}

In addition to these data about neighborhood perception, we also have
block level census data as well as a trove of unaggregated data from
the City of Chicago on the built environment, crimes, 311 reports,
zoning, and the similar. We'd like to set them in relation to each
other.

\section*{Data Details}
The ultimate units of measurement are U.S. Census blocks and the edges
between them. The U.S. Census blocks mainly correspond to city face blocks.
We use a rook adjacency to define the edges, i.e. all edges are between blocks
that share a common border of length greater than 0. In the training data,
there are 5,857 blocks and
  13,887 edges.

For the inter-block similarities, we will use physical, demographic, and
administrative features (Table \ref{tab:Distribution}).

\subsection*{Physical Barriers}
Highways, rail lines, rivers, and major streets often act as neighborhood
boundaries. For each of these types of features, we code an edge feature
variable as 1 if the two adjacent blocks are separated by the feature or code
and 0 otherwise.

We also measure the difference in orientation between blocks. Blocks
are nearly all longer than they are wide, and we calculate the angle
of the longest side. For each pair of blocks, we calculate the
difference between the orientations of the blocks. We normalize the
difference to fall in $[0, 1]$. This measure was inspired by Vivien
Palmer's work on the persistent effect of primary settlements. 

In the late 1920s, Palmer developed the theory that when a subdivision
of a city was originally opened for residential settlement, the
area would start out largely homogenous in its building stock
and population. This homogeneity would lead to the subdivision
maintaining a common fate through the history of the city. The
residents would tend to respond similarly to larger urban processes
and the building stock would tend to have the same patterns of
depreciation and reevaluation.\cite{palmer_primary_1932}

When subdivision were originally platted, the developers tended to
align the streets within their subdivision. Misalignment of blocks are
a rough measure of the where different subdivision meet. 

\subsection*{School Boundaries}
If two adjacent blocks are in different elementary school attendance boundary
then we code an edge feature with 1, 0 otherwise. Similar for high schools.

\subsection*{Demographic Features}
From the U.S. Census, we have block level information on race, age, family
structure, and housing ownership patterns. We can define measures between
blocks for these data.

We will use two measures for the demographic factors. The first is the
absolute difference. The second is the $\chi$ distance $\sqrt{
\frac{1}{2} \sum_i^d \frac{(x_i - y_i)^2}{x_i + y_i}}$, where $x_i$ is
the frequency of bin $i$ of the histogram $\mathbf{x}$ and similar for
$y_i$. This is just square root of the familiar $\chi^2$ statistic. By
taking the square root, $\chi$ becomes a metric.\cite{pele_distance_2011}

For race, the distribution is the number people coded by the Census as
``Hispanic or Latino'', ``Not Hispanic or Latino : White alone'', ``Not Hispanic
or Latino : Black Alone'', and ``Not Hispanic or Latino : Asian alone.''

For age, we'll use the absolute difference between the log median age of
neighboring blocks.

For family structure, the distribution is of households with children
versus households without. 

For housing structure, the distribution is of housing units in the
rental market, housing market, or otherwise disposed. We also use the
absolute difference between the log median density of housing units.

Many blocks have no one living in them or only a handful. In such
cases, it makes little sense to compare demographic distributions. If
either adjacent blocks has fewer than 30 persons living in it, we do
not calculate the demographic differences. There are
7,092 edges where we calculate these
demographic similarities.

% latex table generated in R 3.0.1 by xtable 1.7-1 package
% Fri Feb 28 15:59:17 2014
\begin{table}[ht]
\centering
\begin{tabular}{rrrrrrr}
  \hline
 & Mean & Min & 25th Quant. & Median & 75th Quant. & Max \\ 
  \hline
Sufficient Population & 0.51 &  &  &  &  &  \\ 
  Rail & 0.01 &  &  &  &  &  \\ 
  Highway & 0.02 &  &  &  &  &  \\ 
  Water & 0.01 &  &  &  &  &  \\ 
  Elementary School & 0.13 &  &  &  &  &  \\ 
  High School & 0.03 &  &  &  &  &  \\ 
  Grid Street & 0.22 &  &  &  &  &  \\ 
  Block Angle & 0.20 & 0.00 & 0.00 & 0.00 & 0.50 & 1.00 \\ 
  Log Housing Density & 0.51 & 0.00 & 0.16 & 0.36 & 0.70 & 4.00 \\ 
  Log Median Age & 0.14 & 0.00 & 0.04 & 0.09 & 0.17 & 1.55 \\ 
  Ethnicity & 0.24 & 0.01 & 0.16 & 0.22 & 0.30 & 0.94 \\ 
  Housing & 0.23 & 0.00 & 0.12 & 0.19 & 0.31 & 0.93 \\ 
  Family & 0.13 & 0.00 & 0.05 & 0.11 & 0.19 & 0.81 \\ 
   \hline
\end{tabular}
\caption{Distribution of Barriers and Distances} 
\label{tab:Distribution}
\end{table}



\begin{figure}
\begin{knitrout}
\definecolor{shadecolor}{rgb}{0.969, 0.969, 0.969}\color{fgcolor}

{\centering \includegraphics[width=\maxwidth]{/home/fgregg/sweave-cache/figs/railImage} 

}



\end{knitrout}

\caption{Rail Lines}
\end{figure}

\begin{figure}
\begin{knitrout}
\definecolor{shadecolor}{rgb}{0.969, 0.969, 0.969}\color{fgcolor}

{\centering \includegraphics[width=\maxwidth]{/home/fgregg/sweave-cache/figs/highwayImage} 

}



\end{knitrout}

\caption{Highways}
\end{figure}

\begin{figure}
\begin{knitrout}
\definecolor{shadecolor}{rgb}{0.969, 0.969, 0.969}\color{fgcolor}

{\centering \includegraphics[width=\maxwidth]{/home/fgregg/sweave-cache/figs/gridImage} 

}



\end{knitrout}

\caption{Major Streets}
\end{figure}

\begin{figure}
\begin{knitrout}
\definecolor{shadecolor}{rgb}{0.969, 0.969, 0.969}\color{fgcolor}

{\centering \includegraphics[width=\maxwidth]{/home/fgregg/sweave-cache/figs/waterImage} 

}



\end{knitrout}

\caption{River}
\end{figure}

\begin{figure}
\begin{knitrout}
\definecolor{shadecolor}{rgb}{0.969, 0.969, 0.969}\color{fgcolor}

{\centering \includegraphics[width=\maxwidth]{/home/fgregg/sweave-cache/figs/elementaryImage} 

}



\end{knitrout}

\caption{Elementary School Attendance Boundaries}
\end{figure}

\begin{figure}
\begin{knitrout}
\definecolor{shadecolor}{rgb}{0.969, 0.969, 0.969}\color{fgcolor}

{\centering \includegraphics[width=\maxwidth]{/home/fgregg/sweave-cache/figs/highschoolImage} 

}



\end{knitrout}

\caption{High School Attendance Boundaries}
\end{figure}

\begin{figure}
\begin{knitrout}
\definecolor{shadecolor}{rgb}{0.969, 0.969, 0.969}\color{fgcolor}

{\centering \includegraphics[width=\maxwidth]{/home/fgregg/sweave-cache/figs/sufficientImage} 

}



\end{knitrout}

\caption{Sufficient Population}
\end{figure}


\begin{figure}
\begin{knitrout}
\definecolor{shadecolor}{rgb}{0.969, 0.969, 0.969}\color{fgcolor}

{\centering \includegraphics[width=\maxwidth]{/home/fgregg/sweave-cache/figs/blockAngleImage} 

}



\end{knitrout}

\caption{Difference in Block Orientation}
\end{figure}

\begin{figure}
\begin{knitrout}
\definecolor{shadecolor}{rgb}{0.969, 0.969, 0.969}\color{fgcolor}

{\centering \includegraphics[width=\maxwidth]{/home/fgregg/sweave-cache/figs/_raceImage} 

}



\end{knitrout}

\caption{Difference in Distribution of Race and Ethnicity}
\end{figure}


\begin{figure}
\begin{knitrout}
\definecolor{shadecolor}{rgb}{0.969, 0.969, 0.969}\color{fgcolor}

{\centering \includegraphics[width=\maxwidth]{/home/fgregg/sweave-cache/figs/_ageImage} 

}



\end{knitrout}

\caption{Difference in Log Median Age}
\end{figure}


\begin{figure}
\begin{knitrout}
\definecolor{shadecolor}{rgb}{0.969, 0.969, 0.969}\color{fgcolor}

{\centering \includegraphics[width=\maxwidth]{/home/fgregg/sweave-cache/figs/_housingDiffImage} 

}



\end{knitrout}

\caption{Difference in Log Density of Housing Units}
\end{figure}


\begin{figure}
\begin{knitrout}
\definecolor{shadecolor}{rgb}{0.969, 0.969, 0.969}\color{fgcolor}

{\centering \includegraphics[width=\maxwidth]{/home/fgregg/sweave-cache/figs/_familyImage} 

}



\end{knitrout}

\caption{Difference in Distribution of Family Type}
\end{figure}


\begin{figure}
\begin{knitrout}
\definecolor{shadecolor}{rgb}{0.969, 0.969, 0.969}\color{fgcolor}

{\centering \includegraphics[width=\maxwidth]{/home/fgregg/sweave-cache/figs/_housingImage} 

}



\end{knitrout}

\caption{Difference in Distribution of Housing Type}
\end{figure}

\section*{Modeling}
Since many census blocks are empty, the demographic similarity between
these blocks is not well defined. We might be tempted to ignore these
cases, but deleting these cases would change the topology of the
city. Removing unpopulated blocks would turn populated blocks into
islands separated by commercial corridors. This would not allow for a
neighborhood to span a highway or industrial corridor, a possibility
we do not want to preclude.

We deal with this variation by using dummy variables to effectively
learned two model at once: a nonpopulated and populated block model:

Our overall scoring function is 
\begin{align}
&\operatorname{E}(\mathbf{y}, \mathbf{s}, \mathbf{w}) = \sum_{<i
    j>}^{\mathcal{N}}\epsilon_{i,j}(y_i, y_j, \mathbf{s}_{i,j}, \mathbf{w})  
\end{align}

where 
\begin{equation}
\epsilon_{i,j}(y_i, y_j, \mathbf{s}_{i,j}, \mathbf{w}) = \begin{cases}
    0 \quad\quad\quad y_i = y_j \\
    \phi(\mathbf{s}_{i,j}, \mathbf{w}) \quad y_i \neq y_j \\
  \end{cases}
\end{equation}

and the unpopulated block model is 
\begin{align}
\phi(\mathbf{s}_{i,j}, \mathbf{w}) = & w_0 
                                     + w_1\text{Rail}_{i,j} 
                                     + w_2\text{Water}_{i,j} 
                                     + w_3\text{Highway}_{i,j} \\
                                     &+ w_4\text{Major Street}_{i,j} 
                                     + w_5\text{Elementary School}_{i,j}\\ 
                                     & + w_6\text{High School}_{i,j}
                                     + w_7\text{Block Angle}_{i,j} \\
\end{align}

While the model for the populated neighboring blocks will be

\begin{align}
\phi(\mathbf{s}_{i,j}, \mathbf{w}) = & w_8 
                                     + w_9\text{Rail}_{i,j} 
                                     + w_{10}\text{Water}_{i,j} 
                                     + w_{11}\text{Highway}_{i,j}\\
                                     &+ w_{12}\text{Major Street}_{i,j} 
                                     + w_{13}\text{Elementary School}_{i,j}\\
                                     &+ w_{14}\text{High School}_{i,j} 
                                     + w_{15}\text{Block Angle}_{i,j}\\
                                     &+ w_{16}\text{Family Structure}_{i,j}
                                     + w_{17}\text{Race and Ethnicity}_{i,j}\\
                                     &+ w_{18}\text{Age Structure}_{i,j}  
                                     + w_{19}\text{Housing Structure}_{i,j} \\
                                     &+ w_{20}\text{Housing Density}_{i,j}  
\end{align}

We can combine these models by creating a dummy variable that takes a
value of 1 if neighboring block have sufficient population to support
the demographic distance measures, and 0 if the neighboring blocks do
not. We'll interact this dummy with the populated model and add the
variables to our unpopulated model.\footnote{Actually, our model is
  a little more complicated because we are not just comparing
  groups of people, but also groups of households and groups of
  housing units. Sometimes there is sufficient population to compare
  race, but not sufficient households to compare family structure. We
  use an additional dummy variables to handle these cases}

\begin{align}
\phi(\mathbf{s}_{i,j}, \mathbf{w}) =  & w_0 
                                     + w_1\text{Rail}_{i,j} 
                                     + w_2\text{Water}_{i,j} 
                                     + w_3\text{Highway}_{i,j} \\
                                     &+ w_4\text{Major Street}_{i,j} 
                                     + w_5\text{Elementary School}_{i,j}\\ 
                                     & + w_6\text{High School}_{i,j}
                                     + w_7\text{Block Angle}_{i,j} \\
                                     &+ \text{Populated Blocks}_{i,j}\cdot\\
                                     &\quad (w_8
                                     + w_{9}\text{Rail}_{i,j} 
                                     + w_{10}\text{Water}_{i,j}\\ 
                                     &\quad+ w_{11}\text{Highway}_{i,j}
                                     + w_{12}\text{Major Street}_{i,j}\\ 
                                     &\quad + w_{13}\text{Elementary School}_{i,j} 
                                     + w_{14}\text{High School}_{i,j}\\ 
                                     &\quad + w_{15}\text{Block Angle}_{i,j}
                                     + w_{16}\text{Family Structure}_{i,j}\\
                                     &\quad + w_{17}\text{Race and Ethnicity}_{i,j}
                                     + w_{18}\text{Age Structure}_{i,j}\\  
                                     &\quad+ w_{19}\text{Housing Structure}_{i,j}
                                     + w_{20}\text{Housing Density}_{i,j})\\
\end{align}


\section*{Results}
After training the model using the PyStruct structure learning
framework,\cite{mueller_pystruct:_????} we get two classes of results:
predictions of Chicago neighborhoods and parameter
estimates.\footnote{See \url{https://github.com/fgregg/pystruct} for
  custom model}

Given a set of learned weights we can find, approximately, a lowest
scoring assignment of labels to neighborhoods of the city. However,
some care must be taken to not immediately assign meaning to these
labelings. Two blocks that have been assigned that same label do not
necessarily belong to the same neighborhood. The labelings are only
locally meaningful in that they distinguish a particular neighborhood
from it’s neighbors. Instead, a neighborhood will be a set of block
that have the same label and which are connected components.  

Let’s start by looking at predictions for our training blocks to
highlight some characteristics of our model (Figure
\ref{fig:predictTraining}).

\begin{figure}
\begin{knitrout}
\definecolor{shadecolor}{rgb}{0.969, 0.969, 0.969}\color{fgcolor}

{\centering \includegraphics[width=\maxwidth]{/home/fgregg/sweave-cache/figs/plotPredictions} 

}



\end{knitrout}

\caption{Predicted Neighborhoods}
\label{fig:predictTraining}
\end{figure}

First, our model depends upon C, the normalizer.  C. We choose C by
estimating the model for various values of C between 1.0 and 0.0001
and choosing the value that maximizes the fit between the predicted
neighborhoods and the training neighborhoods. The value of C that gave
the best fit was 0.0005.

Second, our model predicts many, very small neighborhoods. In Figure
\ref{fig:predictTraining}, I have colored the
106 neighborhoods that have ten or more
blocks and grayed out the 980 remaining,
small neighborhoods. There are 594
neighborhoods that consist of a single block.

Third, our model seems to do an adequate job in capturing neighborhoods!!!

\emph{I know I need some sort of measurement here}

Now, turning to the parameters, we'll focus on the parameters for the
populated blocks (Table \ref{tab:parameters}). A positive parameter
means that, all else equal, the total score will be lower if adjacent
blocks are allocated to separate neighborhoods. Negative parameters,
means that all else equal, adjacent blocks should be allocated to the
same neighborhood. So, for example the negative intercepts mean that
adjacent blocks that are identical will ‘prefer’ to be assigned to the
same neighborhood.

Nearly all parameters for barriers and administrative zones have a
positive sign, which is what we expect: dissimilar blocks and blocks
separated by barriers are more apt to be placed in separate
neighborhoods. The negative `attractive' parameter for major street is
likely due to fact that every one of our training neighborhoods
contains multiple major grid streets.

The demographic distances are more complicated. Differences in ethnic
and racial composition is strongly positive. Such a `divisive'
parameter is required to have face validity in Chicago. Differences in
the density of housing units also has the expected sign. Differences
in median age and housing structure are unexpectedly weak, but also
very close to 0. The relatively strong `attractive' parameter for
differences in family structure is very surprising. According the this
model, a block that is all families and a block that has no families
are very apt to belong the same neighborhood. 

\section*{Discussion}
I now feel very good about the technical performance of the
model. Need rewrite with this optimism.



% latex table generated in R 3.0.1 by xtable 1.7-1 package
% Fri Feb 28 16:06:47 2014
\begin{table}[ht]
\centering
\begin{tabular}{rr}
  \hline
 &  \\ 
  \hline
(Intercept) & -0.0026 \\ 
  Railroad & 0.0104 \\ 
  Water & 0.0029 \\ 
  Highway & 0.0064 \\ 
  Major Street & -0.0021 \\ 
  Elementary School & 0.0005 \\ 
  High School & 0.0043 \\ 
  Block Angle & 0.0021 \\ 
  Family Structure & -0.0021 \\ 
  Race and Ethnicity & 0.0040 \\ 
  Age Structure & -0.0008 \\ 
  Density of Housing Units & 0.0037 \\ 
  Housing Structure & -0.0004 \\ 
   \hline
\end{tabular}
\caption{Parameter Estimates for Populated Blocks} 
\label{tab:parameters}
\end{table}





\bibliographystyle{plain}
\bibliography{qp}

\begin{figure}
\includegraphics{../code/training/predicted_chicago_neighborhoods.pdf}
\caption{Predicted Neighborhoods in Chicago}
\end{figure}


\end{document}
