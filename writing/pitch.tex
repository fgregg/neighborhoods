% Created 2011-04-06 Wed 19:46
\documentclass[12pt]{article}
\usepackage{fullpage}
\begin{document}

\title{Pitch}
\author{Forest Timothy Gregg}
\date{06 April 2011}

Social scientists are often frustrated in their investigation of
social, spatial processes because the only data we have have been
aggregated to large, geographic areas with boundaries not of our
choosing, such as census blocks or postal zipcodes. Such aggregations
limit our ability to make reliable inferences about spatial
processes within a given areal unit, and introduce substantial error
if we combine data with different boundaries. While the
ideal solution to such problems would be to use unagreggated,
individual-level data instead, but this is often not possible. Either
the data no longer exist, they are too expensive, or it cannot be
released for privacy reasons. However, when it may not be feasible to
use unaggreagated data, recent advances in spatial statistics, the
accumulation of accurate digital maps, and the progress on the problem
of ecological inference have likely made it possible to statistically
disaggregate data to a level where many of the current challenges
disappear. Disaggregating the outcoming 2010 Census, we would
dramatically demonstrate that possibility and create a dataset that
would be used by thousands of researchers over the next ten years.

In order to disaggregate a census tract, we need outside information
about how we should expect the population to be distributed within the
tract. If we had a map of building footprints, we could
allocate the population only to the parts of the census tract
where a building existed and not to parks, lakes, or
cemeteries. Unfortunately, while such an allocation is substantially
more accurate than assuming a uniform distribution of population over
a tract area, there are a infinite number of ways that you could
allocate the population to the building
footprints. Finding the best allocation ends up being the ecological
inference problem, a classically hard problem.

Fortunately, recent work on this problem by Kosuke Imai, Ying
Lu, and Aaron Strauss seems to indicate that  we can use prior
beliefs about the individual level distributions to make accurate
ecological inferences, and for census data, we have good priors
beliefs. We have individual-level data from the Public-Use Microdata
Samples and the American Housing Surveys. For many municipalities, we
can acquire tax maps with property boundaries, size and number of
floors of the building, and valuation. We have street maps, maps
of how realtors partition a city, school districts, and
parish boundaries. We have maps.

The social sciences also have over a hundred years of theory about the
social processes that distribute individuals and groups across
space. For example, sociology has three big theories that predict
different patterns of racial clustering. First, that cost of housing
is spatially correlated, people live in the most expensive housing
they can afford, and since wealth and race are correlated, different
races live in different areas. Second, individuals have some level of
preference to live amongst those of the same race as themselves, all
else being equal. Third, search processes produce segregation either
because knowledge of housing options is determined by one's already
segregated social network, or realtors steer potential renters and
home-owners towards certain neighborhoods.

While having disjoint theories makes for exciting polemic, it should
also make for better prediction if they make unique predictions within
a blended model.  The hierarchically nested structure of the Census
data affords us the opportunity to train and validate such a model,
and thus, I believe, lay the ground for the next ten years of research
into social, spatial processes.


\end{document}