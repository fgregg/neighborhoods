\section*{Neighborhood Model Representation}
Let's imagine a very small city consisting of four blocks. We can
represent this tiny town as a network where each block is connected
to the blocks that share a street segment as a block face. In this
representation, blocks that are kitty corner are not directly
connected. We'll index the blocks as $1$, $2$, $3$, and $4$.

\begin{figure}[h]
\centering
\tikz{
\draw[help lines] (0,0) grid (2,2);
\node at (0.5, 0.5) {3} ;
\node at (1.5, 1.5) {2} ;
\node at (0.5, 1.5) {1} ;
\node at (1.5, 0.5) {4} ;
}
\end{figure}

\begin{figure}[h]
\centering

\tikz{ %
  \node[latent] (1) {$1$} ; %
  \node[latent, below left=of 1] (2) {$2$} ; %
  \node[latent, below right=of 1] (3) {$3$} ; %
  \node[latent, below left=of 3] (4) {$4$} ; %
  \edge[-] {2,3} {1} ; %
  \edge[-] {2,3} {4} ; %
}

\end{figure}

Suppose that, in our city, there are two neighborhoods. Each block
belongs to either one or the other of these neighborhoods. Neighboring
blocks that are similar are more apt to belong to the same
neighborhood and neighboring blocks that are different are more apt to
belong to different neighborhoods. 

We will denote the neighborhood that the $i$th block belongs to as
$y_i$, and the similarity between blocks $i$ and $j$ as $\phi_{i,j}$.

\begin{figure}[!h]
\centering

\tikz{ %
  \node[latent] (1) {$y_1$} ; %
  \node[latent, below left=of 1] (2) {$y_2$} ; %
  \node[latent, below right=of 1] (3) {$y_3$} ; %
  \node[latent, below left=of 3] (4) {$y_4$} ; %
  \factor[below left=of 1] {1-2} {$\phi_{1,2}$} {} {} ;
  \factor[below right=of 1] {1-3} {$\phi_{1,3}$} {} {} ;
  \factor[below right=of 2] {2-4} {$\phi_{2,4}$} {} {} ;
  \factor[below left=of 3] {3-4} {$\phi_{3,4}$} {} {} ;
  \factoredge[-] {1} {1-2} {2} ; %
  \factoredge[-] {1} {1-3} {3} ; %
  \factoredge[-] {2} {2-4} {4} ; %
  \factoredge[-] {3} {3-4} {4} ; %
  %\edge[-] {2,3} {4} ; %
}

\end{figure}

We want similar, neighboring blocks to belong to the same
neighborhood. One way to formalize this desire is to score every possible
assignment of neighborhoods in such a way that our preferred patterns
have the best score.

Denote a particular pattern of assignment of blocks to neighborhoods
as $\mathbf{y}$.  The score of $\mathbf{y}$ will be
$\operatorname{E}(\mathbf{y})$\footnote{$\operatorname{E}$ as in
  energy not expectation} which will take the following form:

\begin{align}
\operatorname{E}(\mathbf{y}) = \sum_{<i,j>}^{\mathcal{N}}\epsilon_{i,j}(y_i,y_j,\phi_{i,j})
\end{align}

Where $<i,j>$ are indices of a pair of neighboring blocks and such
that $i < j$.  $\mathcal{N}$ is the set of all such pairs of indices.
Also, where

\begin{equation}
\epsilon_{i,j}(y_i,y_j,\phi_{i,j}) = \begin{cases}
  0 &y_i = y_j \\
  \phi_{i,j} &y_i \neq y_j
\end{cases}
\end{equation}

Suppose that we want our preferred neighborhood assignment to have the
lowest score. If our similarity measures $\phi_{i,j}$ are positive
when blocks are similar and negative when blocks are different, then
we will encourage similar, neighboring blocks to belong to the same
neighborhood. Let our city have the following $\phi$'s.

\begin{align*}
&\phi_{1,2} = 1 \\
&\phi_{1,3} = -1 \\
&\phi_{2,4} = -1 \\
&\phi_{3,4} = 1
\end{align*} 

Then, the assignments with the lowest scores are the ones that put
blocks $1$ and $2$ in one neighborhood and $3$ and $4$ in the other
neighborhood, just as we wished (Table \ref{table:lowest}, Table
\ref{table:energy}).

\begin{table}
\centering
  \begin{tabular}{cc}
      \tikz{ %
        \node[latent] (1) {$y_1$} ; %
        \node[latent, below left=of 1] (2) {$y_2$} ; %
        \node[latent, fill=black, below right=of 1] (3) {\textcolor{white}{$y_3$}} ; %
        \node[latent, fill=black, below left=of 3] (4) {\textcolor{white}{$y_4$}} ; %
        \factor[below left=of 1] {1-2} {$1$} {} {} ;
        \factor[below right=of 1] {1-3} {$-1$} {} {} ;
        \factor[below right=of 2] {2-4} {$-1$} {} {} ;
        \factor[below left=of 3] {3-4} {$1$} {} {} ;
        \factoredge[-] {1} {1-2} {2} ; %
        \factoredge[-] {1} {1-3} {3} ; %
        \factoredge[-] {2} {2-4} {4} ; %
        \factoredge[-] {3} {3-4} {4} ; %
      } 
    &
      \tikz{ %
        \node[latent, fill=black] (1) {\textcolor{white}{$y_1$}} ; %
        \node[latent, fill=black, below left=of 1] (2) {\textcolor{white}{$y_2$}} ; %
        \node[latent, below right=of 1] (3) {$y_3$} ; %
        \node[latent, below left=of 3] (4) {$y_4$} ; %
        \factor[below left=of 1] {1-2} {$1$} {} {} ;
        \factor[below right=of 1] {1-3} {$-1$} {} {} ;
        \factor[below right=of 2] {2-4} {$-1$} {} {} ;
        \factor[below left=of 3] {3-4} {$1$} {} {} ;
        \factoredge[-] {1} {1-2} {2} ; %
        \factoredge[-] {1} {1-3} {3} ; %
        \factoredge[-] {2} {2-4} {4} ; %
        \factoredge[-] {3} {3-4} {4} ; %
      } 
    \\
  \end{tabular}
  \caption{Preferred Assignment: If $y_i = 0$, the block is colored
    white. If $y_i = 1$, the block is black.}
  \label{table:lowest}
\end{table}

\begin{table}[h]
\begin{tabular}{cl}
  \\
  $\operatorname{E}(\mathbf{y}) = -2$
  & 
  \begin{tabular}{cccc}
    \scalebox{0.5}{
      \tikz{ %
        \node[latent] (1) {$0$} ; %
        \node[latent, below left=of 1] (2) {$0$} ; %
        \node[latent, fill=black, below right=of 1] (3) {\textcolor{white}{$1$}} ; %
        \node[latent, fill=black, below left=of 3] (4) {\textcolor{white}{$1$}} ; %
        \factor[below left=of 1] {1-2} {} {} {} ;
        \factor[below right=of 1] {1-3} {} {} {} ;
        \factor[below right=of 2] {2-4} {} {} {} ;
        \factor[below left=of 3] {3-4} {} {} {} ;
        \factoredge[-] {1} {1-2} {2} ; %
        \factoredge[-] {1} {1-3} {3} ; %
        \factoredge[-] {2} {2-4} {4} ; %
        \factoredge[-] {3} {3-4} {4} ; %
        %\edge[-] {2,3} {4} ; %
      } 
    }
    &
    \scalebox{0.5}{
      \tikz{ %
        \node[latent, fill=black] (1) {\textcolor{white}{$8$}} ; %
        \node[latent, fill=black, below left=of 1] (2) {\textcolor{white}{$1$}} ; %
        \node[latent, below right=of 1] (3) {$0$} ; %
        \node[latent, below left=of 3] (4) {$0$} ; %
        \factor[below left=of 1] {1-2} {} {} {} ;
        \factor[below right=of 1] {1-3} {} {} {} ;
        \factor[below right=of 2] {2-4} {} {} {} ;
        \factor[below left=of 3] {3-4} {} {} {} ;
        \factoredge[-] {1} {1-2} {2} ; %
        \factoredge[-] {1} {1-3} {3} ; %
        \factoredge[-] {2} {2-4} {4} ; %
        \factoredge[-] {3} {3-4} {4} ; %
        %\edge[-] {2,3} {4} ; %
      } 
    }
    \\
  \end{tabular}
  \\ \hline

  $\operatorname{E}(\mathbf{y}) = 0$
  &
  \begin{tabular}{cccc}
    \scalebox{0.5}{
      \tikz{ %
        \node[latent, fill=black] (1) {\textcolor{white}{$1$}} ; %
        \node[latent, fill=black, below left=of 1] (2) {\textcolor{white}{$1$}} ; %
        \node[latent, fill=black, below right=of 1] (3) {\textcolor{white}{$1$}} ; %
        \node[latent, fill=black, below left=of 3] (4) {\textcolor{white}{$1$}} ; %
        \factor[below left=of 1] {1-2} {} {} {} ;
        \factor[below right=of 1] {1-3} {} {} {} ;
        \factor[below right=of 2] {2-4} {} {} {} ;
        \factor[below left=of 3] {3-4} {} {} {} ;
        \factoredge[-] {1} {1-2} {2} ; %
        \factoredge[-] {1} {1-3} {3} ; %
        \factoredge[-] {2} {2-4} {4} ; %
        \factoredge[-] {3} {3-4} {4} ; %
        %\edge[-] {2,3} {4} ; %
      } 
    }
    \\
    \scalebox{0.5}{
      \tikz{ %
        \node[latent] (1) {$6$} ; %
        \node[latent, fill=black, below left=of 1] (2) {\textcolor{white}{$1$}} ; %
        \node[latent, fill=black, below right=of 1] (3) {\textcolor{white}{$1$}} ; %
        \node[latent, fill=black, below left=of 3] (4) {\textcolor{white}{$1$}} ; %
        \factor[below left=of 1] {1-2} {} {} {} ;
        \factor[below right=of 1] {1-3} {} {} {} ;
        \factor[below right=of 2] {2-4} {} {} {} ;
        \factor[below left=of 3] {3-4} {} {} {} ;
        \factoredge[-] {1} {1-2} {2} ; %
        \factoredge[-] {1} {1-3} {3} ; %
        \factoredge[-] {2} {2-4} {4} ; %
        \factoredge[-] {3} {3-4} {4} ; %
        %\edge[-] {2,3} {4} ; %
      } 
    }
    &
    \scalebox{0.5}{
      \tikz{ %
        \node[latent, fill=black] (1) {\textcolor{white}{$1$}} ; %
        \node[latent, below left=of 1] (2) {$0$} ; %
        \node[latent, fill=black, below right=of 1] (3) {\textcolor{white}{$1$}} ; %
        \node[latent, fill=black, below left=of 3] (4) {\textcolor{white}{$1$}} ; %
        \factor[below left=of 1] {1-2} {} {} {} ;
        \factor[below right=of 1] {1-3} {} {} {} ;
        \factor[below right=of 2] {2-4} {} {} {} ;
        \factor[below left=of 3] {3-4} {} {} {} ;
        \factoredge[-] {1} {1-2} {2} ; %
        \factoredge[-] {1} {1-3} {3} ; %
        \factoredge[-] {2} {2-4} {4} ; %
        \factoredge[-] {3} {3-4} {4} ; %
        %\edge[-] {2,3} {4} ; %
      } 
    }
    &
    \scalebox{0.5}{
      \tikz{ %
        \node[latent, fill=black] (1) {\textcolor{white}{$1$}} ; %
        \node[latent, fill=black, below left=of 1] (2) {\textcolor{white}{$1$}} ; %
        \node[latent, fill=black, below right=of 1] (3) {\textcolor{white}{$1$}} ; %
        \node[latent, below left=of 3] (4) {$0$} ; %
        \factor[below left=of 1] {1-2} {} {} {} ;
        \factor[below right=of 1] {1-3} {} {} {} ;
        \factor[below right=of 2] {2-4} {} {} {} ;
        \factor[below left=of 3] {3-4} {} {} {} ;
        \factoredge[-] {1} {1-2} {2} ; %
        \factoredge[-] {1} {1-3} {3} ; %
        \factoredge[-] {2} {2-4} {4} ; %
        \factoredge[-] {3} {3-4} {4} ; %
        %\edge[-] {2,3} {4} ; %
      }
    }
    &
    \scalebox{0.5}{
      \tikz{ %
        \node[latent, fill=black] (1) {\textcolor{white}{$1$}} ; %
        \node[latent, fill=black, below left=of 1] (2) {\textcolor{white}{$1$}} ; %
        \node[latent, below right=of 1] (3) {$0$} ; %
        \node[latent, fill=black, below left=of 3] (4) {\textcolor{white}{$1$}} ; %
        \factor[below left=of 1] {1-2} {} {} {} ;
        \factor[below right=of 1] {1-3} {} {} {} ;
        \factor[below right=of 2] {2-4} {} {} {} ;
        \factor[below left=of 3] {3-4} {} {} {} ;
        \factoredge[-] {1} {1-2} {2} ; %
        \factoredge[-] {1} {1-3} {3} ; %
        \factoredge[-] {2} {2-4} {4} ; %
        \factoredge[-] {3} {3-4} {4} ; %
        %\edge[-] {2,3} {4} ; %
      } 
    }
    \\
    \scalebox{0.5}{
      \tikz{ %
        \node[latent] (1) {$0$} ; %
        \node[latent, fill=black, below left=of 1] (2) {\textcolor{white}{$1$}} ; %
        \node[latent, fill=black, below right=of 1] (3) {\textcolor{white}{$1$}} ; %
        \node[latent, below left=of 3] (4) {$0$} ; %
        \factor[below left=of 1] {1-2} {} {} {} ;
        \factor[below right=of 1] {1-3} {} {} {} ;
        \factor[below right=of 2] {2-4} {} {} {} ;
        \factor[below left=of 3] {3-4} {} {} {} ;
        \factoredge[-] {1} {1-2} {2} ; %
        \factoredge[-] {1} {1-3} {3} ; %
        \factoredge[-] {2} {2-4} {4} ; %
        \factoredge[-] {3} {3-4} {4} ; %
        %\edge[-] {2,3} {4} ; %
      }
    }
    &
    \scalebox{0.5}{
      \tikz{ %
        \node[latent, fill=black] (1) {\textcolor{white}{$1$}} ; %
        \node[latent, below left=of 1] (2) {$0$} ; %
        \node[latent, below right=of 1] (3) {$0$} ; %
        \node[latent, fill=black, below left=of 3] (4) {\textcolor{white}{$1$}} ; %
        \factor[below left=of 1] {1-2} {} {} {} ;
        \factor[below right=of 1] {1-3} {} {} {} ;
        \factor[below right=of 2] {2-4} {} {} {} ;
        \factor[below left=of 3] {3-4} {} {} {} ;
        \factoredge[-] {1} {1-2} {2} ; %
        \factoredge[-] {1} {1-3} {3} ; %
        \factoredge[-] {2} {2-4} {4} ; %
        \factoredge[-] {3} {3-4} {4} ; %
        %\edge[-] {2,3} {4} ; %
      } 
    }
    \\
    \scalebox{0.5}{
      \tikz{ %
        \node[latent, fill=black] (1) {\textcolor{white}{$1$}} ; %
        \node[latent, below left=of 1] (2) {$0$} ; %
        \node[latent, below right=of 1] (3) {$0$} ; %
        \node[latent, below left=of 3] (4) {$0$} ; %
        \factor[below left=of 1] {1-2} {} {} {} ;
        \factor[below right=of 1] {1-3} {} {} {} ;
        \factor[below right=of 2] {2-4} {} {} {} ;
        \factor[below left=of 3] {3-4} {} {} {} ;
        \factoredge[-] {1} {1-2} {2} ; %
        \factoredge[-] {1} {1-3} {3} ; %
        \factoredge[-] {2} {2-4} {4} ; %
        \factoredge[-] {3} {3-4} {4} ; %
        %\edge[-] {2,3} {4} ; %
      }
    }
    &
    \scalebox{0.5}{
      \tikz{ %
        \node[latent] (1) {$0$} ; %
        \node[latent, fill=black, below left=of 1] (2) {\textcolor{white}{$1$}} ; %
        \node[latent, below right=of 1] (3) {$0$} ; %
        \node[latent, below left=of 3] (4) {$0$} ; %
        \factor[below left=of 1] {1-2} {} {} {} ;
        \factor[below right=of 1] {1-3} {} {} {} ;
        \factor[below right=of 2] {2-4} {} {} {} ;
        \factor[below left=of 3] {3-4} {} {} {} ;
        \factoredge[-] {1} {1-2} {2} ; %
        \factoredge[-] {1} {1-3} {3} ; %
        \factoredge[-] {2} {2-4} {4} ; %
        \factoredge[-] {3} {3-4} {4} ; %
        %\edge[-] {2,3} {4} ; %
      } 
    }
    &
    \scalebox{0.5}{
      \tikz{ %
        \node[latent] (1) {$0$} ; %
        \node[latent, below left=of 1] (2) {$0$} ; %
        \node[latent, below right=of 1] (3) {$0$} ; %
        \node[latent, fill=black, below left=of 3] (4) {\textcolor{white}{$1$}} ; %
        \factor[below left=of 1] {1-2} {} {} {} ;
        \factor[below right=of 1] {1-3} {} {} {} ;
        \factor[below right=of 2] {2-4} {} {} {} ;
        \factor[below left=of 3] {3-4} {} {} {} ;
        \factoredge[-] {1} {1-2} {2} ; %
        \factoredge[-] {1} {1-3} {3} ; %
        \factoredge[-] {2} {2-4} {4} ; %
        \factoredge[-] {3} {3-4} {4} ; %
        %\edge[-] {2,3} {4} ; %
      } 
    }
    &
    \scalebox{0.5}{
      \tikz{ %
        \node[latent] (1) {$0$} ; %
        \node[latent, below left=of 1] (2) {$0$} ; %
        \node[latent, fill=black, below right=of 1] (3) {\textcolor{white}{$1$}} ; %
        \node[latent, below left=of 3] (4) {$0$} ; %
        \factor[below left=of 1] {1-2} {} {} {} ;
        \factor[below right=of 1] {1-3} {} {} {} ;
        \factor[below right=of 2] {2-4} {} {} {} ;
        \factor[below left=of 3] {3-4} {} {} {} ;
        \factoredge[-] {1} {1-2} {2} ; %
        \factoredge[-] {1} {1-3} {3} ; %
        \factoredge[-] {2} {2-4} {4} ; %
        \factoredge[-] {3} {3-4} {4} ; %
        %\edge[-] {2,3} {4} ; %
      } 
    }
    \\
    % All Zeros %
    \scalebox{0.5}{
      \tikz{ %
        \node[latent] (1) {$0$} ; %
        \node[latent, below left=of 1] (2) {$0$} ; %
        \node[latent, below right=of 1] (3) {$0$} ; %
        \node[latent, below left=of 3] (4) {$0$} ; %
        \factor[below left=of 1] {1-2} {} {} {} ;
        \factor[below right=of 1] {1-3} {} {} {} ;
        \factor[below right=of 2] {2-4} {} {} {} ;
        \factor[below left=of 3] {3-4} {} {} {} ;
        \factoredge[-] {1} {1-2} {2} ; %
        \factoredge[-] {1} {1-3} {3} ; %
        \factoredge[-] {2} {2-4} {4} ; %
        \factoredge[-] {3} {3-4} {4} ; %
        %\edge[-] {2,3} {4} ; %
      } 
    }
    \\
    \end{tabular}
  \\ \hline
  % Mismatched %
  $\operatorname{E}(\mathbf{y}) = 2$
  &
  \begin{tabular}{cccc}
    \scalebox{0.5}{
      \tikz{ %
        \node[latent] (1) {$0$} ; %
        \node[latent, fill=black, below left=of 1] (2) {\textcolor{white}{$1$}} ; %
        \node[latent, below right=of 1] (3) {$0$} ; %
        \node[latent, fill=black, below left=of 3] (4) {\textcolor{white}{$1$}} ; %
        \factor[below left=of 1] {1-2} {} {} {} ;
        \factor[below right=of 1] {1-3} {} {} {} ;
        \factor[below right=of 2] {2-4} {} {} {} ;
        \factor[below left=of 3] {3-4} {} {} {} ;
        \factoredge[-] {1} {1-2} {2} ; %
        \factoredge[-] {1} {1-3} {3} ; %
        \factoredge[-] {2} {2-4} {4} ; %
        \factoredge[-] {3} {3-4} {4} ; %
        %\edge[-] {2,3} {4} ; %
      }
    }
    &
    \scalebox{0.5}{
      \tikz{ %
        \node[latent, fill=black] (1) {\textcolor{white}{$1$}} ; %
        \node[latent, below left=of 1] (2) {$0$} ; %
        \node[latent, fill=black, below right=of 1] (3) {\textcolor{white}{$1$}} ; %
        \node[latent, below left=of 3] (4) {$0$} ; %
        \factor[below left=of 1] {1-2} {} {} {} ;
        \factor[below right=of 1] {1-3} {} {} {} ;
        \factor[below right=of 2] {2-4} {} {} {} ;
        \factor[below left=of 3] {3-4} {} {} {} ;
        \factoredge[-] {1} {1-2} {2} ; %
        \factoredge[-] {1} {1-3} {3} ; %
        \factoredge[-] {2} {2-4} {4} ; %
        \factoredge[-] {3} {3-4} {4} ; %
        %\edge[-] {2,3} {4} ; %
      }
    }
    \\
  \end{tabular}
\end{tabular}

\caption{Scores of Neighborhood Assignments}
\label{table:energy}
\end{table}

For small networks, we can find these best scoring assignments by
checking all possible patterns of assignments. Unfortunately, this
becomes quickly impractical for larger cities. With two possible
neighborhoods, the number of possible assignments is $2^N$ where N is
the number of blocks. This permutational explosion means that for even
small cities, we cannot possibly check every possible neighborhood
assignment in human-scale time. For the same reason, traditional
Markov Chain Monte Carlo methods for finding the lowest scoring
assignment also prove intractable.\footnote{In order to converge on
  the mode of the distribution of scores, we have to calculate a
  normalizing constant, which is the sum of the scores of all possible
  assignments. This is computationally too expensive. There have been
  a number of attempts to find an acceptable substitute for the
  normalizing constant, but the empirical results have
  disappointed.\cite{li_mrf_2009}}

\subsection*{Graph Methods}
Fortunately, researchers, largely in the field of computer vision,
have developed methods to quickly find the lowest scoring assignment
for problems like ours. This work traces back to a 1986 paper by
Greig, Porteous, and Sehult, where they demonstrated that finding
lowest scoring assignment for a two neighborhood case was equivalent
to solving the problem of finding the minimum cut of a
graph.\cite{greig_exact_1989}

In that paper, the authors set up an scoring problem that is slightly
different than ours. In addition to terms that depended upon pairs of
neighbors, their score also included terms for individual nodes.

\begin{align}
\operatorname{E}(\mathbf{y}) = \sum_i\epsilon_i(y_i) + \sum_{<i
  j>}^{\mathcal{N}}\epsilon_{i,j}(y_i,y_j,\phi_{i,j})
\end{align}

\noindent
The node terms were of one of two forms, either

\begin{equation}
\epsilon_{i}(y_i) = \begin{cases}
  0 &y_i = 0 \\
  \phi_{i} \geq 0 &y_i = 1
\end{cases}
\end{equation}

\noindent
or 

\begin{equation}
\epsilon_{i}(y_i) = \begin{cases}
  \phi_{i} \geq 0 &y_i = 0 \\
  0 &y_i = 1
\end{cases}
\end{equation}

\noindent
They had the same form of $\epsilon_{i,j}$

\begin{equation}
\epsilon_{i,j}(y_i,y_j,\phi_{i,j}) = \begin{cases}
  0 &y_i = y_j \\
  \phi_{i,j} &y_i \neq y_j
\end{cases}
\end{equation}

\noindent
but unlike us, $\phi_{i,j} \geq 0$ for all $<i, j>$ in $\mathcal{N}$. 

One of their networks could look like this

\begin{figure}[!h]
\centering

\tikz{ %
  \node[latent] (1) {$y_1$} ; %
  \node[latent, below left=of 1] (2) {$y_2$} ; %
  \node[latent, below right=of 1] (3) {$y_3$} ; %
  \node[latent, below left=of 3] (4) {$y_4$} ; %
  \factor[below left=of 1] {1-2} {$1$} {} {} ;
  \factor[below right=of 1] {1-3} {$1$} {} {} ;
  \factor[below right=of 2] {2-4} {$1$} {} {} ;
  \factor[below left=of 3] {3-4} {$1$} {} {} ;
  \factoredge[-] {1} {1-2} {2} ; %
  \factoredge[-] {1} {1-3} {3} ; %
  \factoredge[-] {2} {2-4} {4} ; %
  \factoredge[-] {3} {3-4} {4} ; %
  %\edge[-] {2,3} {4} ; %
}

\end{figure}

Based on this undirected network, Greig and his co-authors constructed
a special, directed network where every edge has a cost associated
with it. For every variable $y_i$ in the undirected network, the
directed network contains a node $z_i$.  For every edge $(y_i, y_j)$ in the
undirected graph, we introduce a directed edge from $z_i$ to $z_j$ and
another from $z_j$ to $z_i$ with associated costs of $\phi_{i,j}$.

In addition to the original nodes, we also introduce two special
nodes: a source node called $s$ and a target, or sink, node called
$t$. If $\epsilon_i(0) = 0$ then there would be a directed edge from
$s$ to the $i$th node. If $\epsilon_i(1) = 0$ then there would be
directed edge from the $i$th node to $t$.

Suppose that $\epsilon_1(1)=2$, $\epsilon_2(0)=1$, $\epsilon_3(1)=3$,
and $\epsilon_4(0)=3$. The corresponding directed graph is shown in
Figure \ref{fig:directed}. 

\begin{figure}[!h]
\centering

\begin{tikzpicture}[auto, >= stealth', shorten >= 1pt, node
    distance=2cm, thick]
\tikzset{vertex/.style = {shape=circle, draw, minimum size = 1.5em}}
\tikzset{Dedge/.style = {->}}
\tikzset{Uedge/.style = {<->}}

\node[vertex, above=of 1] (0) {$s$} ; %
\node[vertex] (1) {$y_1$} ; %
\node[vertex, below left=of 1] (2) {$y_2$} ; %
\node[vertex, below right=of 1] (3) {$y_3$} ; %
\node[vertex, below left=of 3] (4) {$y_4$} ; %
\node[vertex, below=of 4] (5) {$t$} ; %
\path
  (0) edge [Dedge] node {2} (1)
  (0) edge [Dedge, bend left] node[right] {3} (3)
  (1) edge [Uedge] node {1} (2) 
  (1) edge [Uedge] node {1} (3) 
  (2) edge [Uedge] node {1} (4) 
  (3) edge [Uedge] node {1} (4) 
  (2) edge [Dedge, bend right] node[left] {1} (5)
  (4) edge [Dedge] node {3} (5)  ;

\end{tikzpicture}
\caption{S-T Network}
\label{fig:directed}
\end{figure}

In this type of network, an s-t cut is a set of edges that if removed,
or cut, would result in the original network being split into disjoint
subneworks, one which included the source node and the other the
target node. The minimal cut problem is finding the s-t cut where the
sum of the associated costs of the cut edges is minimal. In our
example, removing the edges $(z_2, t)$ and $(z_4, t)$ is an s-t cut,
but not the minimum cut. The minimum cut would be removing edges
$(z_1, z_2)$ and $(z_3, z_4)$.

Every s-t cut can be mapped onto a pattern of neighborhood block
assignments by letting $y_i = 1$ if node $z_i$ ends up in the $t$
subnetwork and $y_i = 0$ otherwise.

Greig and his coauthors demonstrated that the minimum cut of the $s-t$
directed graph always corresponded to the minimum scoring pattern of
block assignment. Because computer scientists have known how to solve
the min cut problem swiftly since the 1950s, this mean that a large
class of network configurations problems were now tractable \cite{ford_maximal_1956}.

Greig's original result only applied to scoring functions where there
where blocks could belong to only one of two classes and where the
scoring function had important property called submodularity.  A
submodular scoring function would be one where $\epsilon_{i,j}(0,0) +
\epsilon_{i,j}(1,1) \leq \epsilon_{i,j}(0,1) + \epsilon_{i,j}(1,0)$
for all $i,j$ in $\mathcal{N}$. Our scoring function is decisively not
submodular because we want it to cost more to put together very
dissimilar neighboring blocks than to assign them to them to different
neighborhoods. That is, for very dissimilar blocks we want it to hold
that $\epsilon_{i,j}(0,0) + \epsilon_{i,j}(1,1) \boldsymbol{>}
\epsilon_{i,j}(0,1) + \epsilon_{i,j}(1,0)$.

Since we have submodular scoring functions and more than two
neighborhood names, we can't use the minimum cuts algorithm
directly. However, since the 1980s there have been a number of
extensions that that allow us to find approximate solutions to
multi-label non-submodular scoring problems like ours \cite{something}.

\section*{Learning similarities}
In the previous section, we showed how we could construct a scoring
function so that a preferred pattern of block assignments would have
the best score. Now we will turn to how, given a preferred assignment,
we can learn such a scoring function. 

We will first make our lives easier by only considering simple, linear
similarity terms.. As before, the scoring function will be

\begin{align}
\operatorname{E}(\mathbf{y}) = \sum_{<i,j>}^{\mathcal{N}}\epsilon_{i,j}(y_i,y_j,\phi_{i,j})
\end{align}

\noindent
and

\begin{equation}
\epsilon_{i,j}(y_i,y_j,\phi_{i,j}) = \begin{cases}
  0 &y_i = y_j \\
  \phi_{i,j} &y_i \neq y_j
\end{cases}
\end{equation}

\noindent
the similarity term $\phi_{i,j}$ will be the weighted sum of
observed similarities between block $i$ and $j$. 

\begin{align}
\phi_{i,j} = w_0 + w_1s_{1,i,j} + w_2s_{2,i,j} + ... + w_ns_{n,i,j}
\end{align} 

\noindent
An observed inter-block similarity could be absolute difference in
population, or dummy variable indicating whether the blocks are
separated by a railroad, or similar.

\noindent

With this form, learning a scoring function means finding the weights
$\mathbf{w}: \{w_0, w_1, ..., w_{m-1}, w_m\}$ that give a better score
to our preferred pattern of neighborhood assignments than any other
pattern.

As stated, this learning task is under defined. There will not be only
one set of weights that will give our preferred pattern a lower score
than any other pattern. If there is single set of weights that does
so, there will be an infinite number of other sets of weights that
will do the same. In our four block city, our preferred
neighborhood assignment had the best score with block-by-block
similarities of 1, -1, -1, 1. That assignment would also have had the
best score if the similarities had been 2, -2, -2, 2 or 1000, -1500,
-2000, 4000.

In order to specify a unique set of weights, we need to add two
constraints. First, observe that if a set of weights gives the
preferred assignment a lower score than any other assignment, there is
some difference between the score of the preferred assignment and the
score of the next best scoring assignment. Call this difference the
'margin. The first condition is that we seek the set weights that give
our preferred assignment the largest margin. Formally,

\begin{align}
&\argmax_{\mathbf{w}} \mathbf{\gamma} \\
&\text{such that} \\
&\operatorname{E}(\mathbf{y}, \mathbf{s}, \mathbf{w})
- \operatorname{E}(\mathbf{y}^*, \mathbf{s}, \mathbf{w}) \geq \gamma\\ 
\end{align}

This still does not specify a unique set of weights. To do that, we
constrain the sizes of the weights. We could do this by
limiting the sum of the absolute values of the weights. A similar, but
more mathematically convenient choice is to require that
$\sqrt{\sum_i^M w_i^2 = 1}$. 

So, now we can state the problem as 
%
\begin{align*}
&\argmax_{\mathbf{w}:||\mathbf{w}||=1} \mathbf{\gamma} \\
&\text{such that} \\
&\operatorname{E}(\mathbf{y}, \mathbf{s}, \mathbf{w})
- \operatorname{E}(\mathbf{y}^*, \mathbf{s}, \mathbf{w}) \geq \gamma\\ 
&\text{for all } \mathbf{y} \text{ where } \mathbf{y} \text{ is in the set of
  possible neighborhood assignments}\\
&\text{and } \mathbf{y} \neq \mathbf{y}^*
\end{align*}
%

Here $\mathbf{y}*$ is the target neighborhood assignment and $\mathbf{s}$
are similarity measures between blocks. 

Large margin problems like this one have an equivalent canonical
representation: 
%
\begin{align*}
&\argmin_{\mathbf{w}} \frac{1}{2}||\mathbf{w}||^2 \\
&\text{such that} \\
&\operatorname{E}(\mathbf{y}, \mathbf{s}, \mathbf{w})
- \operatorname{E}(\mathbf{y}^*, \mathbf{s}, \mathbf{w}) \geq 1 \\ 
&\text{for all } \mathbf{y} \text{ where } \mathbf{y} \text{ is in the set of
  possible neighborhood assignments}\\
&\text{and } \mathbf{y} \neq \mathbf{y}^*
\end{align*}

This is a quadratic problem, so .. szummer.

\subsection*{Learning Sketch}
We usually can't directly solve this quadratic program because there
are so many possible assignments, however we can still find the
optimal weights through the following procedure.

Initialize the weights to some starting value. Create an empty set of
constraints $\mathcal{S}$. Then, find the neighborhood assignment that
has the lowest score given those weights. Call this assignment
$\mathbf{y}_1$ and add it to the constraint set $\mathcal{S}$.

Update the weights by solving the quadratic program: 
%
\begin{align*}
&\argmin_{\mathbf{w}} \frac{1}{2}||\mathbf{w}||^2 \\
&\text{such that} \\
&\operatorname{E}(\mathbf{y}, \mathbf{s}, \mathbf{w})
- \operatorname{E}(\mathbf{y}^*, \mathbf{s}, \mathbf{w}) \geq 1 \\ 
&\text{for all } \mathbf{y} \text{ where } \mathbf{y} \text{ is in } \mathcal{S}\\
&\text{and } \mathbf{y} \neq \mathbf{y}^*
\end{align*}
%

Now find the lowest energy neighborhood assignment given the updated
weights. Call this assignment $\mathbf{y}_2$ and add it to the
constraint set $\mathcal{S}$. Continue until the lowest energy
assignment given the weights is either the target assignment or
already in the constraint set.\cite{szummer_learning_2008}


\subsection*{Empirical Loss}
Some readers may have noticed that the our four-city block example
still does not have a unique solution for the learning problem we have
set up. There are two different neighborhood assignments that have the
same lowest score, and they both achieve our goal of putting similar
blocks in the same neighborhood and dissimilar blocks in different
neighborhoods (Table \ref{table:lowest}).

\begin{table}
\centering
  \begin{tabular}{cc}
      \tikz{ %
        \node[latent] (1) {$0$} ; %
        \node[latent, below left=of 1] (2) {$0$} ; %
        \node[latent, fill=black, below right=of 1] (3) {\textcolor{white}{$1$}} ; %
        \node[latent, fill=black, below left=of 3] (4) {\textcolor{white}{$1$}} ; %
        \factor[below left=of 1] {1-2} {} {} {} ;
        \factor[below right=of 1] {1-3} {} {} {} ;
        \factor[below right=of 2] {2-4} {} {} {} ;
        \factor[below left=of 3] {3-4} {} {} {} ;
        \factoredge[-] {1} {1-2} {2} ; %
        \factoredge[-] {1} {1-3} {3} ; %
        \factoredge[-] {2} {2-4} {4} ; %
        \factoredge[-] {3} {3-4} {4} ; %
      } 
    &
      \tikz{ %
        \node[latent, fill=black] (1) {\textcolor{white}{$1$}} ; %
        \node[latent, fill=black, below left=of 1] (2) {\textcolor{white}{$1$}} ; %
        \node[latent, below right=of 1] (3) {$0$} ; %
        \node[latent, below left=of 3] (4) {$0$} ; %
        \factor[below left=of 1] {1-2} {} {} {} ;
        \factor[below right=of 1] {1-3} {} {} {} ;
        \factor[below right=of 2] {2-4} {} {} {} ;
        \factor[below left=of 3] {3-4} {} {} {} ;
        \factoredge[-] {1} {1-2} {2} ; %
        \factoredge[-] {1} {1-3} {3} ; %
        \factoredge[-] {2} {2-4} {4} ; %
        \factoredge[-] {3} {3-4} {4} ; %
      } 
    \\
  \end{tabular}
  \caption{Equivalent Lowest Assignments}
  \label{table:lowest}
\end{table}

Happily, and for unrelated reasons, practicioners have found that good
performance for this type of learning problem depends upon incorporating an
empirical loss function into the objective function. If we follow
suit, we will also side step this problem of multiple equivalent
assignments. 

Below we show how to incorporate the empirical loss:
%
\begin{align*}
&\argmin_{\mathbf{w}} \frac{1}{2}||\mathbf{w}||^2 +
  C\cdot\xi\\
&\text{such that} \\
&\operatorname{E}(\mathbf{y}, \mathbf{s}, \mathbf{w})
- \operatorname{E}(\mathbf{y}^*, \mathbf{s}, \mathbf{w}) \geq
\Delta(\mathbf{y}^*, \mathbf{y}) - \xi\\ 
& \xi \geq 0\\
&\text{for all } \mathbf{y} \text{ where } \mathbf{y} \text{ is in the set of
  possible neighborhood assignments,}\\
&\xi \text{ is a slack variable, and } \Delta \text{ is an empirical loss.}\\
\end{align*}
%

The key to solving this problem is finding assignments that have the
lowest score, where the score incorporates the empirical loss: i.e.
%
\begin{align}
\operatorname{E}(\mathbf{y}) = \sum_{<i j>}^{\mathcal{N}}\epsilon_{i,j}(y_i,y_j) + C\cdot\Delta(\mathbf{y}*,
\mathbf{y})
\end{align}
%
If the loss can be decomposed over the blocks like
%
\begin{align}
\operatorname{E}(\mathbf{y}) = \sum_{<i j>}^{\mathcal{N}}\epsilon_{i,j}(y_i,y_j) + C\cdot\sum_i^N\Delta(y_i*, y_i)
\end{align}
%
then the objective has the form that allows us to use an algorithm
like QPBO to quickly find an approximate minimum assignment. If the
loss function does not decompose we have to check an exponentially
many possible assignment, typically an intractable problem.

We want a loss function that increases the further we are from our
target assignment. A natural choice is to penalize every block that is
incorrectly assigned. If we penalize every error the same then we have
the classic Hamming loss. 

$\sum_i^N\begin{cases}
  0 &y_i^* = y_i \\
  1 &y_i^* \neq y_i
\end{cases}$. 

Incorporating this empirical loss into our learning problem means
that, of a set of previously equivalent assignments, only the
assignment closest to the training data will be have the lowest score.


\section*{Maybe Ditch}
scalar similarity 
and where 
Let's consider similarity measures that take the following linear form


Where $x_{i,k}$ is the $k$th feature of block $i$. These features can
be scalars or vectors of numbers and could correspond to attributes
like population or distribution of race and ethnicity. A function
$\operatorname{S}$ is similarity or distance function that maps two
features to a real number. These functions must be symmetric; for each
$\operatorname{S}$ and features $a$ and $b$, $\operatorname{S}(a,b) =
\operatorname{S}(b,a)$.
