\documentclass{article}
\usepackage{amsmath}
\usepackage{url}
\usepackage{endnotes}
\usepackage{tikz}
\usepackage{subfigure}
\title{Memo}
\author{Forest Gregg}

\begin{document}
\maketitle

On Wednesday, April 17, 2013, we discussed setting up the training of
a MRF model for segmenting cities into neighborhoods. 

From scraping Craigslist data, I have a large collections of points
which are claimed to belong to particular neighborhoods. Estimating a
separate KDE for each neighborhood, I can estimate a decision
boundaries between neighborhoods. I take these decisions boundaries as
the ``ground truth'' which I will use to train a model that segments a
city.

I had been thinking about a model where the nodes potentials
correspond to blocks and the edge potentials are a function of linear
combination of weighted similarities between blocks, where a
similarity might be the difference between reported crimes, or some
other attribute of a block.  However, it was not clear to me how to
assign node potentials, since I didn't want to recover the names of
neighborhoods, only the arbitrary segmentation.

You suggested a dual model, where the nodes correspond to the edges
between between blocks. The node potential is still a function of the
similarity between blocks, but instead of assigning an arbitrary
neighborhood name, the node is assigned to either be an ``boundary''
or not. The edge potentials would then be smoothing function that
would encourage contiguous boundaries and discourage some perverse
boundaries patterns.


\subsection*{Follow Up}

You said that you would ask one of your students if he would be
willing to give me some guidance in setting up a structured SVM
learning.

We discussed various loss functions, and you said you would send me a
recent paper of yours related to this topic.

\subsection*{Other Notes}
For the inference step, I told you I have been using GCO library
\url{http://vision.csd.uwo.ca/code/}. You said this looked like a good
choice.


\end{document}

